%
%  使用 xelatex 编译
%
\documentclass[UTF8]{ctexart}
\usepackage[a4paper,top=2.54cm,bottom=2.54cm,left=3.17cm,right=3.17cm,%
            includehead,includefoot]{geometry}

%%%%%===== 字体设置
%\setmainfont{Adobe Garamond Pro} % 衬线字体 roman \rmfamily
%\setsansfont{Myriad Pro} % 无衬线字体 sans serif \sffamily
\setmonofont{Consolas}   % 等宽字体 typewriter \ttfamily

%%%%% ===== 设置章节
\ctexset{%
  contentsname={目\ \ 录},
  listfigurename={插\ 图\ 目\ 录},
  listtablename={表\ 格\ 目\ 录},
  bibname={参\ 考\ 文\ 献},
  section={name={第,节},
           number=\chinese{section},
           format=\centering,
           nameformat={\large\bfseries},
           titleformat={\large\bfseries}
           },
  subsection={format=\raggedright,
           nameformat={\bfseries},
           titleformat={\bfseries}
           }
}

%%%%%===== 宏包调用
\usepackage{amsmath,amssymb,amsfonts} % 数学宏包
\usepackage{bm} % 数学宏包
\usepackage{cases} % 数学宏包
\usepackage{yhmath} % 数学宏包
\usepackage{graphicx,subfigure} % 插图宏包
\usepackage{epstopdf} % eps转pdf宏包
\usepackage{xcolor,float} % 颜色与浮动对象宏包
\usepackage[xetex,pagebackref,breaklinks,colorlinks]{hyperref} % 超链接宏包
\hypersetup{citecolor=blue,         % 引用标记颜色
            linkcolor=blue,         % 内部普通链接的颜色
            urlcolor=blue,          % url 链接的颜色
            bookmarksnumbered=true, % 书签带章节编号
            bookmarksopen=true     % 书签目录展开
           }
\usepackage{booktabs} % 表格宏包
\usepackage{fancyvrb} % 摘录宏包
  \fvset{formatcom=\color{blue},frame=single,rulecolor=\color{red}}
\usepackage[numbers,square,sort&compress]{natbib} % 参考文献


%%%%% ===== 定理环境
\usepackage[amsmath,thref,thmmarks,hyperref]{ntheorem} % 定理宏包
\theorempreskipamount1em % spacing before the environment
\theorempostskipamount0em  % spacing after the environment
\theoremstyle{plain}
\theoremheaderfont{\normalfont\heiti}
\theorembodyfont{\normalfont\kaishu}
\theoremindent0em
\theoremseparator{\hspace{0.2em}}
\theoremnumbering{arabic}
\newtheorem{theorem}{定理}[section]
\newtheorem{property}{性质}[section]
\newtheorem{definition}{定义}[section]
\newtheorem{lemma}{引理}[section]
\newtheorem{remark}{注记}[section]
\newtheorem{corollary}{推论}[section]
\newtheorem{example}{例}[section]
%
\theoremstyle{nonumberplain}
\theorembodyfont{\normalfont}
\theoremsymbol{\ensuremath{\Box}}
\newtheorem{proof}{证明}


%===== 算法与源代码
\usepackage{algorithm,algpseudocode}  % 算法
\usepackage{listings} % 源代码
\renewcommand{\lstlistlistingname}{源代码目录}
\renewcommand{\lstlistingname}{源代码}
\lstset{language=Matlab}
\lstset{escapechar=`}
\lstset{basicstyle=\ttfamily\small,showstringspaces=false,tabsize=2}
\lstset{flexiblecolumns=true}
\lstset{xleftmargin=1ex,xrightmargin=1ex}
\lstset{frame=tblr,frameround=tttt}  %单线, 圆角框
%%\lstset{frame=TBLR}  %双线方框
%\lstset{frame=shadowbox,rulesepcolor=\color{blue}}
\lstset{commentstyle=\color{red},keywordstyle=\color{blue},caption=\lstname,%
        breaklines=true,backgroundcolor=\color{lightgray!20}}
\lstset{numbers=left, numberstyle=\small, stepnumber=1, numbersep=1em}

%%%%%===== 页眉页脚 =====================================================
\usepackage{fancyhdr}
\pagestyle{fancy}
\fancyhf{}
\renewcommand{\headrulewidth}{0pt}
\renewcommand{\sectionmark}[1]{\markboth{\uppercase{#1}}{}}
\chead{\leftmark}
\cfoot{\thepage}

%%%%% ===== 其他设置
\numberwithin{equation}{section} % 数学公式编号方式
\allowdisplaybreaks  % 允许在长公式中换页
\linespread{1.2}  % 基本行间距

%%%%% ===== 自定义命令
\renewcommand{\C}{\ensuremath{\mathbb{C}}}
\newcommand{\R}{\ensuremath{\mathbb{R}}}
\newcommand{\A}{\mathcal{A}}
\newcommand{\lam}{\lambda}
%
\DeclareMathOperator{\tridiag}{tridiag} % 算子


%%%%% ===== 论文开始 =====================================================
\begin{document}

\title{复对称线性方程组的广义修正 HSS 迭代方法}

\author{XXX \thanks{email@ecnu.edu.cn, 本文由某某某基金资助.} \\[1em]
  华东师范大学数学系\\
  上海, 200241, 中国}

\date{}

\maketitle

\begin{abstract}
复系数线性方程组广泛存在于科学与工程计算的众多应用领域中.
目前求解大规模稀疏线性方程组的主要方法是迭代法,
预处理方法是改善迭代法收敛性和稳定性的主要技术.
事实上, 好的预处理子已成为迭代法取得成功的关键因素.
\end{abstract}

\section{引言}
考虑复系数线性方程组
\begin{equation}\label{eq:problem}
  \mathcal{A}x \equiv (W+iT)x=b
\end{equation}
其中 $W \in \R^{n \times n}$ 对称正定,
$T \in \R^{n \times n}$ 对称半正定, $x, b \in \C^{n}$.
此时 $\mathcal{A}$ 的 Hermitian 与 skew-Hermitian 部分分别为
$$
  \mathcal{H}=\frac12(\mathcal{A}+\mathcal{A}^{*})=W, \qquad \mathcal{S}=\frac12(\mathcal{A}-\mathcal{A}^{*})=iT,
$$
即 HSS \cite{BGN-03} 算法中的 $\mathcal{H}$ 和 $\mathcal{S}$ 分别为 $W$ 和 $iT$.
关于这类问题的研究可参见
\cite{BBC-10,BBC-11,NMP-84,VM-90}

\subsection{MHSS 和 PMHSS 迭代方法}
为了避免在迭代过程中求解复系数线性方程组,
Bai, Benzi, Chen 针对 HSS 方法进行了适当地修正,
提出了 MHSS \cite{BBC-10} 算法和 PMHSS \cite{BBC-11} 算法.
PMHSS 算法具体迭代格式如下:
\begin{algorithm}[H]
\caption{PMHSS 算法\label{Alg:PMHSS}}
\begin{algorithmic}[1]
  \State 给定一个初值 $ x^{(0)} \in \C^{n} $  和常数 $\alpha>0$
  \For{$k = 1,2, \ldots $ 直到序列 $\{x^{(k)}\}_{k=0}^{\infty}$ 收敛}
  \State 解方程: $(\alpha V+W)x^{(k+\frac{1}{2})}=(\alpha V-iT)x^{(k)}+b $
  \State 解方程: $(\alpha V+T)x^{(k+1)}=(\alpha V+iW)x^{(k+\frac{1}{2})}-ib$
  \EndFor
\end{algorithmic}
\end{algorithm}

在 PMHSS 算法中, 如果取 $V=I$, 则可得到 MHSS 算法.
文献中还证明了对于任意 $\alpha >0$,
MHSS 和 PMHSS 算法都无条件收敛到问题 \eqref{eq:problem} 的唯一的解.
在文献 \cite{BBC-10,BBC-11} 的数值实验中, 我们可以看出,
MHSS 和 PMHSS 方法相较 HSS 和 PHSS, 在迭代步数和计算时间上,
都有一定程度的优势和改进.

以上算法都是针对 $W$ 和 $T$ 是实对称矩阵的情形进行的 HSS 算法的推广.
文献 \cite{GW-12} 考虑了 $W, T$ 非对称的情形,
并给出了 $W$ 和 $T$ 非对称时, MHSS 算法的收敛性.
\begin{theorem}
  设 $A=W+iT$, 其中 $W, T \in \R^{n \times n}$.
  若 $(1-i)W$ 正定, $(1+i)T$ 半正定, 则
  $$
    \rho(M(\alpha))<1, \quad\forall \ \alpha>0.
  $$
\end{theorem}

\section{GMHSS 迭代方法}
我们考虑对 MHSS 算法做进一步的推广. 设
$$
  \mathcal{A}=W+iT,
$$
其中 $W, T$ 都是实对称矩阵.
并假定存在某个实数 $\beta$, 使得 $\beta W+T$ 正定, $W-\beta T$ 半正定.
分别在原线性方程组 \eqref{eq:problem} 两边同时乘以
$(\beta -i)$ 和 $(1+\beta i)$, 可得
$$
  \begin{cases}
    (\beta-i)(W+iT)x=(\beta-i)b, \\
    (1+\beta i)(W+iT)x=(1+\beta i)b.
  \end{cases}
$$
由此, 我们可以构造出如下的广义修正 HSS 迭代方法, 记为 GMHSS.

\begin{algorithm}[H]
\caption{GMHSS 算法\label{Alg:GMHSS}}
\begin{algorithmic}[1]
  \State 给定一个初值 $ x^{(0)} \in \C^{n} $ 和常数 $\alpha>0$
  \For{$k = 1,2, \ldots $ 直到序列 $\{x^{(k)}\}_{k=0}^{\infty}$ 收敛}
  \State 解方程:
    $(\alpha I+\beta W+T)x^{(k+\frac{1}{2})}
     =(\alpha I+iW-i\beta T)x^{(k)}+(\beta -i)b$
  \State 解方程:
    $(\alpha I+W-\beta T)x^{(k+1)}
      =(\alpha I-i\beta W-iT)x^{(k+\frac{1}{2})}+(1+i \beta)b$
  \EndFor
\end{algorithmic}
\end{algorithm}

根据假设, $\beta W+T \in \R^{n \times n}$ 正定,
$W-\beta T \in \R^{n \times n}$ 半正定,
所以当 $\alpha>0$ 时, $\alpha I+\beta W+T$ 和
$\alpha I+W-\beta T$ 都是实对称正定矩阵.
因此在 GMHSS 算法的两步交替迭代中,
需要求解的子线性方程组的系数矩阵都是对称正定的,
因此可以用不完全 Cholesky 分解或者不精确的共轭梯度法来计算.

\subsection{GMHSS 算法收敛性分析}
我们首先给出可以一个描述一般两步分裂迭代算法收敛的准则 \cite{BGN-03}.
\begin{lemma} \label{lemma1}
  设 $A \in \C^{n \times n}$, $A=M_{1}-N_{1}=M_{2}-N_{2}$ 是矩阵 $A$ 的两个分裂,
  给定初值 $ x^{(0)} \in \C^{n} $, 对于序列 $\{x^{k}\}(k = 0,1,2...)$
  并通过以下的迭代方法求解 $x^{(k+1)}$:
  $$
    \begin{cases}
      M_{1}x^{(k+\frac{1}{2})}=N_{1}x^{(k)}+b\\
      M_{2}x^{(k+1)}=N_{2}x^{(k+\frac{1}{2})}+b
    \end{cases}
  $$
  则对于 $\forall x^{(0)} \in C^{n}$, 迭代序列 ${x^{(k)}}$ 均收敛到
  线性方程组 $Ax=b$ 的唯一解 $x_{*} \in C^{n}$.
\end{lemma}

我们通过直接计算可知 GMHSS 方法的迭代格式为
\begin{equation}\label{GMHSS}
  x^{(k+1)}=M(\alpha)x^{(k)}+G(\alpha)b,\quad k=0,1,2,....
\end{equation}
利用引理 \ref{lemma1}, 我们可以得到以下定理.
\begin{theorem}\label{Th:GMHSS-convergence}
  设 $\A=W+iT \in \C^{n \times n}$, $W \in \R^{n \times n}$, $T \in \R^{n \times n}$,
  并且存在某个实数 $\beta$, 使得 $\beta W+T$ 对称正定, $W-\beta T$ 对称半正定.
  那么, 对于 $\forall \alpha > 0$, 我们有
  $$ \rho(M(\alpha)) \leq \sigma(\alpha). $$
\end{theorem}
\begin{proof}
  此处省略若干字 ... ...
\end{proof}

下面的结论给出了近似最优参数 $\alpha_*$ 的选取方法.

\begin{corollary}
  设上述定理中的条件都满足, 矩阵 $\beta W+T$
  的最小和最大特征根分别记为 $\lam_{min}$ 和 $\lam_{max}$,
  则
  $$
    \alpha_{*} \triangleq \arg\min_{\alpha}
      \max_{\lam_{min} \leq \lam \leq \lam_{max}} \sigma(\alpha)
    =\sqrt{\lam_{min}\lam_{max}}.
  $$
\end{corollary}


\section{数值算例}
在本节中, 我们使用文献 \cite{BBC-10} 中的两个数值算例来测试
GMHSS 算法的数值效果.

\begin{example} \label{example11}
  考虑线性方程组
  $$
    \left[\left(K+\frac{3-\sqrt{3}}{\tau} I\right)
     + i\left(K+\frac{3+\sqrt{3}}{\tau} I\right)\right]x=b,
  $$
  其中 $\tau$ 表示时间步长,
  该问题来源于抛物方程 $R_{22}$-Pad\'{e} 近似的中心差分离散,
  更多的细节可参见文献 \cite{AK-00,BBC-10,BBC-11}.
\end{example}

表格 \ref{Tab:11} 中列出的是 MHSS 和 GMHSS 算法的数值结果.

\begin{table}[H]
\begin{center}
\caption{MHSS 和 GMHSS 数值结果比较\label{Tab:11}}\smallskip
\begin{tabular}{|c|c|c|c|c|c|c|} \hline
  \multicolumn{2}{|c|}{$n$}& $2^8$ &$2^{10}$ &$2^{12}$ &$2^{14}$ &$2^{16}$  \\ \hline
      & $\alpha_{*}$ & 1.16  & 0.78  & 0.55  & 0.40    & 0.30     \\ \cline{2-7}
 MHSS & IT           &  39   &  53   &  72   &  98     & 133      \\ \cline{2-7}
      & CPU          &2.25e-2&7.36e-2& 1.02e+0& 1.46e+1 & 1.55e+2  \\ \hline
      & $\alpha_{*}$ & 0.23  & 0.11  & 0.06  & 0.03    & 0.01     \\ \cline{2-7}
GMHSS & IT           &  36   &  37   &  39   &  40     & 41       \\ \cline{2-7}
      & CPU          &1.36e-2&3.76e-2&3.16e-1&  2.58e+0 & 2.20e+1  \\ \hline
\end{tabular}
\end{center}
\end{table}

从表 \ref{Tab:11} 中可以看出,
无论是在迭代步数上, 还是在迭代时间上, GMHSS 算法都优于 MHSS 算法.


%%%%% ===== 参考文献
\begin{thebibliography}{99}

\bibitem{AK-00}
O. Axelsson and A. Kucherov,
Real valued iterative methods for solving complex symmetric linear systems,
Numer. Linear Algebra Appl., 7 (2000), 197--218.


\bibitem{BBC-10}
Z-Z Bai, M. Benzi and F. Chen,
Modified HSS iteration methods for a class of complex symmetric linear systems,
Computing, 87 (2010), 93--111.

\bibitem{BBC-11}
Z-Z Bai, M. Benzi and F. Chen,
On preconditioned MHSS iteration methods for complex symmetric linear systems,
Numer. Algorithms, 56 (2011), 297--317.

\bibitem{BGN-03}
Z-Z Bai, G. H. Golub and M. K. Ng,
Hermitian and skew-Hermitian splitting methods for non-Hermitian positive definite linear systems,
SIAM J.Matrix. Anal.Appl., 24 (2003), 603--626.

\bibitem{GW-12}
X-X Guo and S. Wang,
Modified HSS iteration methods for a class of non-Hermitian
positive-definite linear systems,
Applied Mathematics and Computation, 218 (2012), 10122--10128.

\bibitem{LHL-07}
L. Li, T-Z Huang and X-P Liu,
Modified Hermitian and skew-Hermitian splitting methods for non-Hermitian positive-definite linear systems,
Numer.Linear Algebra Appl., 14 (2007), 217--235.

\bibitem{NMP-84}
S. Novikov, S. V. Manakov, L. P. Pitaevski, V. E. Zakharov,
Theory of Solitons. The Inverse Scattering Method,
Translated from Russian. Contemporary Soviet Mathematics. Consultants Bureau[Plenum], New York (1984).

\bibitem{VM-90}
H. van der Vorst, J. Melissen,
A Petrov-Galerkin type method for solving $Ax=b$, where A is symmetric complex,
IEEE Trans Magn., 26 (1990), 706--708.

\bibitem{Wan-12}
王珅,
两类非 Hermitian 线性方程组的迭代法研究, 硕士论文, 中国海洋大学, 2012.

\end{thebibliography}

\end{document}
