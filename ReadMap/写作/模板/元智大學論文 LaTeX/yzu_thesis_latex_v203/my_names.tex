%
% this file is encoded in utf-8
% v2.02 (Sep. 12, 2012)
% 填入你的論文題目、姓名等資料
% 如果題目內有必須以數學模式表示的符號,請用 \mbox{} 包住數學模式,如下範例
% 如果中文名字是單名,與姓氏之間建議以全形空白填入,如下範例
% 英文名字中的稱謂,如 Prof. 以及 Dr.,其句點之後請以不斷行空白~代替一般空白,如下範例
% 如果你的指導教授沒有如預設的三位這麼多,則請把相對應的多餘教授的中文、英文名
%    的定義以空的大括號表示
%    如,\renewcommand\advisorCnameB{}
%          \renewcommand\advisorEnameB{}
%          \renewcommand\advisorCnameC{}
%          \renewcommand\advisorEnameC{}

% 論文題目 (中文)
\renewcommand\cTitle{%
我的碩士論文題目 \mbox{$\cal{H}_\infty$} 與 \mbox{Al$_x$Ga$_{1-x}$As}
}

% 論文題目 (英文)
\renewcommand\eTitle{%
My Thesis Title  \mbox{$\cal{H}_\infty$} and \mbox{Al$_x$Ga$_{1-x}$As}
}

% 我的姓名 (中文)
\renewcommand\myCname{王鐵雄}

% 我的姓名 (英文)
\renewcommand\myEname{Aron Wang}

% 指導教授A的姓名 (中文)
\renewcommand\advisorCnameA{南宮明博士}

% 指導教授A的姓名 (英文)
\renewcommand\advisorEnameA{Dr.~Ming Nangong}

% 指導教授B的姓名 (中文)
\renewcommand\advisorCnameB{李斯坦博士}

% 指導教授B的姓名 (英文)
\renewcommand\advisorEnameB{Dr.~Stein Lee}

% 指導教授C的姓名 (中文)
\renewcommand\advisorCnameC{徐 石博士}

% 指導教授C的姓名 (英文)
\renewcommand\advisorEnameC{Dr.~Sean Hsu}

% 校名 (中文)
\renewcommand\univCname{元智大學}

% 校名 (英文)
\renewcommand\univEname{Yuan Ze University}

% 系所名 (中文)
\renewcommand\deptCname{光電工程研究所}

% 系所全名 (英文)
\renewcommand\fulldeptEname{Department of Photonics Engineering}

% 系所短名 (英文, 用於書名頁學位名領域)
\renewcommand\deptEname{Photonics Engineering}

% 學院英文名 (如無,則以空的大括號表示)
\renewcommand\collEname{College of Electrical and Communication Engineering}

% 學位名 (中文)
\renewcommand\degreeCname{碩士}

% 學位名 (英文)
\renewcommand\degreeEname{Master of Science}

% 口試年份 (中文、民國)
\renewcommand\cYear{一○一}

% 口試月份 (中文)
\renewcommand\cMonth{七} 

% 口試年份 (阿拉伯數字、西元)
\renewcommand\eYear{2012} 

% 口試月份 (英文)
\renewcommand\eMonth{July}

% 學校所在地 (英文)
\renewcommand\ePlace{Chungli, Taoyuan, Taiwan}

%畢業級別;用於書背列印;若無此需要可忽略
\newcommand\GraduationClass{100}

%%%%%%%%%%%%%%%%%%%%%%