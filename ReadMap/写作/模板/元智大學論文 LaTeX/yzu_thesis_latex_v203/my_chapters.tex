%
% this file is encoded in utf-8
% v2.02 (Sep. 12, 2012)
% front matter 前頁
% 包括封面、書名頁、中文摘要、英文摘要、誌謝、目錄、表目錄、圖目錄、符號說明
% 在撰寫各章草稿時,可以把此部份「關掉」,以節省無謂的編譯時間。
% 實際內容由
%    my_names.tex, my_cabstract.tex, my_eabstract.tex, my_ackn.tex, my_symbols.tex
% 決定
% yzu_frontpages.tex 此檔只提供整體架構的定義,不需更動
% 在撰寫各章草稿時,可以把此部份「關掉」,以節省無謂的編譯時間。
%
% this file is encoded in big5
% v2.03 (Dec. 19, 2012)
% do not change the content of this file
% unless the thesis layout rule is changed
% �L���ק糧�ɤ��e�A���D�դ�ק�F
% �ʭ��B�ѦW���B����K�n�B�^��K�n�B�x�¡B�ؿ��B���ؿ��B�ϥؿ��B�Ÿ�����
% �������榡

% make the line spacing in effect
\renewcommand{\baselinestretch}{\mybaselinestretch}
\large % it needs a font size changing command to be effective

% default variables definitions
% ���B�u�O�w�]�ȡA���ݧ�惡�B
% ��� my_names.tex ���e
\newcommand\cTitle{�פ��D��}
\newcommand\eTitle{MY THESIS TITLE}
\newcommand\myCname{���K��}
\newcommand\myEname{Aron Wang}
\newcommand\advisorCnameA{�n�c���դh}
\newcommand\advisorEnameA{Dr.~Ming Nangong}
\newcommand\advisorCnameB{�����Z�դh}
\newcommand\advisorEnameB{Dr.~Stein Lee}
\newcommand\advisorCnameC{�}�@�۳դh}
\newcommand\advisorEnameC{Dr.~Sean~Hsu}
\newcommand\univCname{�����j��}
\newcommand\univEname{Yuan Ze University}
\newcommand\deptCname{���q�u�{��s��}
\newcommand\fulldeptEname{Graduate School of Electro-Optical Engineering}
\newcommand\deptEname{Photonics Engineering}
\newcommand\collEname{College of Engineering}
\newcommand\degreeCname{�Ӥh}
\newcommand\degreeEname{Master of Science}
\newcommand\cYear{�E�Q�|}
\newcommand\cMonth{��}
\newcommand\eYear{2006}
\newcommand\eMonth{June}
\newcommand\ePlace{Chungli, Taoyuan, Taiwan}


 % user's names; to replace those default variable definitions
%
% this file is encoded in utf-8
% v2.0 (Apr. 5, 2009)
% 填入你的論文題目、姓名等資料
% 如果題目內有必須以數學模式表示的符號,請用 \mbox{} 包住數學模式,如下範例
% 如果中文名字是單名,與姓氏之間建議以全形空白填入,如下範例
% 英文名字中的稱謂,如 Prof. 以及 Dr.,其句點之後請以不斷行空白~代替一般空白,如下範例
% 如果你的指導教授沒有如預設的三位這麼多,則請把相對應的多餘教授的中文、英文名
%    的定義以空的大括號表示
%    如,\renewcommand\advisorCnameB{}
%          \renewcommand\advisorEnameB{}
%          \renewcommand\advisorCnameC{}
%          \renewcommand\advisorEnameC{}

% 論文題目 (中文)
\renewcommand\cTitle{%
我的碩士論文題目 \mbox{$\cal{H}_\infty$} 與 \mbox{Al$_x$Ga$_{1-x}$As}
}

% 論文題目 (英文)
\renewcommand\eTitle{%
My Thesis Title  \mbox{$\cal{H}_\infty$} and \mbox{Al$_x$Ga$_{1-x}$As}
}

% 我的姓名 (中文)
\renewcommand\myCname{王鐵雄}

% 我的姓名 (英文)
\renewcommand\myEname{Aron Wang}

% 指導教授A的姓名 (中文)
\renewcommand\advisorCnameA{南宮明博士}

% 指導教授A的姓名 (英文)
\renewcommand\advisorEnameA{Dr.~Ming Nangong}

% 指導教授B的姓名 (中文)
\renewcommand\advisorCnameB{李斯坦博士}

% 指導教授B的姓名 (英文)
\renewcommand\advisorEnameB{Dr.~Stein Lee}

% 指導教授C的姓名 (中文)
\renewcommand\advisorCnameC{徐 石博士}

% 指導教授C的姓名 (英文)
\renewcommand\advisorEnameC{Dr.~Sean Hsu}

% 校名 (中文)
\renewcommand\univCname{元智大學}

% 校名 (英文)
\renewcommand\univEname{Yuan Ze University}

% 系所名 (中文)
\renewcommand\deptCname{光電工程學系}

% 系所全名 (英文)
\renewcommand\fulldeptEname{Department of Electro-Optical Engineering}

% 系所短名 (英文, 用於書名頁學位名領域)
\renewcommand\deptEname{Electro-Optical Engineering}

% 學院英文名 (如無,則以空的大括號表示)
\renewcommand\collEname{College of Electrical and Communication Engineering}

% 學位名 (中文)
\renewcommand\degreeCname{碩士}

% 學位名 (英文)
\renewcommand\degreeEname{Master of Science}

% 口試年份 (中文、民國)
\renewcommand\cYear{九十八}

% 口試月份 (中文)
\renewcommand\cMonth{七} 

% 口試年份 (阿拉伯數字、西元)
\renewcommand\eYear{2009} 

% 口試月份 (英文)
\renewcommand\eMonth{July}

% 學校所在地 (英文)
\renewcommand\ePlace{Chungli, Taoyuan, Taiwan}

%畢業級別;用於書背列印;若無此需要可忽略
\newcommand\GraduationClass{97}

%%%%%%%%%%%%%%%%%%%%%%

% �ϥ� hyperref �b pdf ²����̶�J�������
\ifx\hypersetup\undefined
	\relax  % do nothing
\else
	\hypersetup{
	pdftitle=\cTitle,
	pdfauthor=\myCname}
\fi
	

\newcommand\itsempty{}
%%%%%%%%%%%%%%%%%%%%%%%%%%%%%%%
%       YZU cover �ʭ�
%%%%%%%%%%%%%%%%%%%%%%%%%%%%%%%
%
\begin{titlepage}
% no page number
% next page will be page 1

% aligned to the center of the page
\begin{center}
% font size (relative to 12 pt):
% \large (14pt) < \Large (18pt) < \LARGE (20pt) < \huge (24pt)< \Huge (24 pt)
%
\makebox[6cm][s]{\Huge\univCname}\\  %��ܤ���զW
\vspace{1.5cm}
\makebox[12cm][s]{\Huge\deptCname}\\ %��ܤ���t�ҦW
\vspace{1.5cm}
\makebox[6cm][s]{\Huge\degreeCname �פ�}\\ %��ܽפ���� (����)
\vspace{1.5cm}
%
% Set the line spacing to single for the titles (to compress the lines)
\renewcommand{\baselinestretch}{1}   %��Z 1 ��
%\large % it needs a font size changing command to be effective
\Large\cTitle\\  % �����D��
%
\vspace{1cm}
%
\Large\eTitle\\ %�^���D��
\vspace{2cm}
% \makebox is a text box with specified width;
% option s: stretch; option l: left aligned
% use \makebox to make sure
% �u��s�͡v �P�u���ɱб¡voccupy the same width
% Names are filled in a box with pre-defined width
% the left and right sides of �u�G�voccupy the same width (use \hspace{} to fill the short)
% to guarantee �u�G�vis at the center
% assume the width of a Chinese character is 1.2em
% 4.8em is determined by the length of the longest string "���ɱб�"
% 7.2em is determined by the length of the possibly longest name + title "�ڶ����ӳդh"
\hspace{2.4em}%
\makebox[4.8em][s]{\Large ��s��}%
\makebox[1em][c]{\Large �G}%
\makebox[7.2em][l]{\Large\myCname}\\  % ��ܧ@�̤���W
%
\hspace{2.4em}%
\makebox[4.8em][s]{\Large ���ɱб�}%
\makebox[1em][c]{\Large �G}%
\makebox[7.2em][l]{\Large\advisorCnameA}\\  %��ܫ��ɱб�A����W
%
% �P�_�O�_���@�P���ɪ��б� B
\ifx \advisorCnameB  \itsempty
\relax % �S�� B �б¡A�ҥH���������A���L����ť�
\else
% �@�P���ɪ��б� B
\hspace{2.4em}%
\makebox[4.8em][s]{}%
\makebox[1em][c]{}%
\makebox[7.2em][l]{\Large\advisorCnameB}\\%��ܫ��ɱб�B����W
\fi
%
% �P�_�O�_���@�P���ɪ��б� C
\ifx \advisorCnameC  \itsempty
\relax % �S�� C �б¡A�ҥH���������A���L����ť�
\else
% �@�P���ɪ��б� C
\hspace{2.4em}%
\makebox[4.8em][s]{}%
\makebox[1em][c]{}%
\makebox[7.2em][l]{\Large\advisorCnameC}\\%��ܫ��ɱб�B����W
\fi
%
\vfill
\makebox[10cm][s]{\Large ���إ���\cYear �~\cMonth ��}%
%
\end{center}
% Resume the line spacing to the desired setting
\renewcommand{\baselinestretch}{\mybaselinestretch}   %��_��]�w
% it needs a font size changing command to be effective
% restore the font size to normal
\normalsize
\end{titlepage}
%%%%%%%%%%%%%%

%% �q�K�n�쥻�大�e�������H�p�gù���Ʀr�L���X
% ���O�q�u�ѦW���v(�����L���X) �N�}�l�p��
\setcounter{page}{1}
\pagenumbering{roman}
%%%%%%%%%%%%%%%%%%%%%%%%%%%%%%%
%       �ѦW�� 
%%%%%%%%%%%%%%%%%%%%%%%%%%%%%%%
%
\newpage

% �P�_�O�_�n�B���L�H
\ifx\mywatermark\undefined 
  \thispagestyle{empty}  % �L���X�B�L header (�L�B���L)
\else
  \thispagestyle{EmptyWaterMarkPage} % �L���X�B���B���L
\fi

%no page number
% create an entry in table of contents for �ѦW��
\phantomsection % for hyperref to register this
\addcontentsline{toc}{chapter}{\nameInnerCover}


% aligned to the center of the page
\begin{center}
% font size (relative to 12 pt):
% \large (14pt) < \Large (18pt) < \LARGE (20pt) < \huge (24pt)< \Huge (24 pt)
% Set the line spacing to single for the titles (to compress the lines)
\renewcommand{\baselinestretch}{1}   %��Z 1 ��
% it needs a font size changing command to be effective
%�����D��
\Large\cTitle\\ %%%%%
\vspace{1cm}
% �^���D��
\Large\eTitle\\ %%%%%
%\vspace{1cm}
\vfill
% \makebox is a text box with specified width;
% option s: stretch
% use \makebox to make sure
% �u��s�͡G�v �P�u���ɱб¡G�voccupy the same width
\large %to have correct em value
\makebox[4.8em][s]{��s��}%
\makebox[1em][c]{�G}%
\makebox[7.2em][l]{\myCname}%%%%%
\hfill%
\makebox[2cm][l]{Student:}%
\makebox[5cm][l]{\myEname}\\ %%%%%
%
%\vspace{1cm}
%
\makebox[4.8em][s]{���ɱб�}%
\makebox[1em][c]{�G}%
\makebox[7.2em][l]{\advisorCnameA}%%%%%
\hfill%
\makebox[2cm][l]{Advisor:}%
\makebox[5cm][l]{\advisorEnameA}\\ %%%%%
%
% �P�_�O�_���@�P���ɪ��б� B
\ifx \advisorCnameB  \itsempty
\relax % �S�� B �б¡A�ҥH���������A���L����ť�
\else
%�@�P���ɪ��б�B
\makebox[4.8em][s]{}%
\makebox[1em][c]{}%
\makebox[7.2em][l]{\advisorCnameB}%%%%%
\hfill%
\makebox[2cm][l]{}%
\makebox[5cm][l]{\advisorEnameB}\\ %%%%%
\fi
%
% �P�_�O�_���@�P���ɪ��б� C
\ifx \advisorCnameC  \itsempty
\relax % �S�� C �б¡A�ҥH���������A���L����ť�
\else
%�@�P���ɪ��б�C
\makebox[4.8em][s]{}%
\makebox[1em][c]{}%
\makebox[7.2em][l]{\advisorCnameC}%%%%%
\hfill%
\makebox[2cm][s]{}%
\makebox[5cm][l]{\advisorEnameC}\\ %%%%%
\fi
%
% Resume the line spacing to the desired setting
\renewcommand{\baselinestretch}{\mybaselinestretch}   %��_��]�w
\normalsize %it needs a font size changing command to be effective
\large
%
\vfill
\makebox[4cm][s]{\univCname}\\% �զW
\makebox[6cm][s]{\deptCname}\\% �t�ҦW
\makebox[3cm][s]{\degreeCname �פ�}\\% �Ǧ�W
%
%\vspace{1cm}
\vfill
\large
A Thesis\\%
Submitted to %
%
\fulldeptEname\\%�t�ҥ��W (�^��)
%
%
\ifx \collEname  \itsempty
\relax % �S���ǰ|�W (�^��)�A�ҥH���������A���L����ť�
\else
% ���ǰ|�W (�^��)
\collEname\\% �ǰ|�W (�^��)
\fi
%
\univEname\\%�զW (�^��)
%
in Partial Fulfillment of the Requirements\\
%
for the Degree of\\
%
\degreeEname\\%�Ǧ�W(�^��)
%
in\\
%
\deptEname\\%�t�ҵu�W(�^��F�����Ǧ���)
%
\eMonth\ \eYear\\%��B�~ (�^��)
%
\ePlace% �ǮթҦb�a (�^��)
\vfill
���إ���%
\cYear% %%%%%
�~%
\cMonth% %%%%%
��\\
\end{center}
% restore the font size to normal
\normalsize
\clearpage
%%%%%%%%%%%%%%%%%%%%%%%%%%%%%%%
%       �פ�f�թe���f�w�� (�p���X�A�����L���X) 
%%%%%%%%%%%%%%%%%%%%%%%%%%%%%%%
%
% insert the printed standard form when the thesis is ready to bind
% �b�f�է�����A�A�N�wñ�W���f�w�ѩ�J�H�K�˭q
% create an entry in table of contents for �f�w��
% �ثe�e�X�ťխ�
\newpage%
{\thispagestyle{empty}%
\phantomsection % for hyperref to register this
\addcontentsline{toc}{chapter}{\nameCommitteeForm}%
\mbox{}\clearpage}

%%%%%%%%%%%%%%%%%%%%%%%%%%%%%%%
%       ���v�� (�p���X�A�����L���X) 
%%%%%%%%%%%%%%%%%%%%%%%%%%%%%%%
%
% insert the printed standard form when the thesis is ready to bind
% �b�f�է�����A�A�N�wñ�W�����v�ѩ�J�H�K�˭q
% create an entry in table of contents for ���v��
% �ثe�e�X�ťխ�
% (Oct. 31, 2012 �s�W�w�A���v�Ѧ��|���A�o�ط|�e�X�|�i�ťխ�)
\newpage% 
{\thispagestyle{empty}%
\phantomsection % for hyperref to register this
\addcontentsline{toc}{chapter}{\nameCopyrightForm}%
\mbox{}\clearpage}

\newpage% 2nd 
{\thispagestyle{empty}%
\mbox{}\clearpage}

\newpage% 3rd
{\thispagestyle{empty}%
\mbox{}\clearpage}

\newpage% 4th
{\thispagestyle{empty}%
\mbox{}\clearpage}

%%%%%%%%%%%%%%%%%%%%%%%%%%%%%%%
%       ����K�n 
%%%%%%%%%%%%%%%%%%%%%%%%%%%%%%%
%
\newpage
\thispagestyle{plain}  % �L header�A���b�B���L�Ҧ��U�|���B���L
% create an entry in table of contents for ����K�n
\phantomsection % for hyperref to register this
\addcontentsline{toc}{chapter}{\nameCabstract}

% aligned to the center of the page
\begin{center}
% font size (relative to 12 pt):
% \large (14pt) < \Large (18pt) < \LARGE (20pt) < \huge (24pt)< \Huge (24 pt)
% Set the line spacing to single for the names (to compress the lines)
\renewcommand{\baselinestretch}{1}   %��Z 1 ��
% it needs a font size changing command to be effective
\large\cTitle\\  %�����D��
\vspace{0.83cm}
% \makebox is a text box with specified width;
% option s: stretch
% use \makebox to make sure
% each text field occupies the same width
\makebox[3em][l]{�ǥ͡G}%
\makebox[4.8em][l]{\myCname}%�ǥͤ���m�W
\hfill%
%
\makebox[5em][l]{���ɱб¡G}%
\makebox[7.2em][l]{\advisorCnameA}\\ %�б�A����m�W
%
% �P�_�O�_���@�P���ɪ��б� B
\ifx \advisorCnameB  \itsempty
\relax % �S�� B �б¡A�ҥH���������A���L����ť�
\else
%�@�P���ɪ��б�B
\makebox[3em][l]{}%
\makebox[4.8em][l]{}%%%%%
\hfill%
\makebox[5em][l]{}%
\makebox[7.2em][l]{\advisorCnameB}\\ %�б�B����m�W
\fi
%
% �P�_�O�_���@�P���ɪ��б� C
\ifx \advisorCnameC  \itsempty
\relax % �S�� C �б¡A�ҥH���������A���L����ť�
\else
%�@�P���ɪ��б�C
\makebox[3em][l]{}%
\makebox[4.8em][l]{}%%%%%
\hfill%
\makebox[5em][l]{}%
\makebox[7.2em][l]{\advisorCnameC}\\ %�б�B����m�W
\fi
%
\vspace{0.42cm}
%
\univCname\deptCname\\ %�զW�t�ҦW
\vspace{0.83cm}
%\vfill
\makebox[2.5cm][s]{�K�n}\\
\end{center}
% Resume the line spacing to the desired setting
\renewcommand{\baselinestretch}{\mybaselinestretch}   %��_��]�w
%it needs a font size changing command to be effective
% restore the font size to normal
\normalsize
%%%%%%%%%%%%%
�K�n���e��������s�ت��A��ƨӷ��A��s��k�ε��G���A�� 500--1000 �r�A�åH�@�������C�n���K�n���ӧΦp�F�|�A�W�U�Ҽe�A�߸y�ֲӡC����s���ʾ��B�ت����y�z�A�P��s���G�i�઺�v�T�P���Τ��y�z�A���h�󤤶�����s����y�z�C��Ū�̤@���Y��T�w���פ�O�_����M�䪺�Ъ��C�@�G�T�|�����C�K�E�A�@�G�T�|�����C�K�E�A�@�G�T�|�����C�K�E�A�@�G�T�|�����C�K�E�A�@�G�T�|�����C�K�E�A�@�G�T�|�����C�K�E�A�@�G�T�|�����C�K�E�A�@�G�T�|�����C�K�E�A�@�G�T�|�����C�K�E�A�@�G�T�|�����C�K�E�C�@�G�T�|�����C�K�E�A�@�G�T�|�����C�K�E�A�@�G�T�|�����C�K�E�A�@�G�T�|�����C�K�E�A�@�G�T�|�����C�K�E�A�@�G�T�|�����C�K�E�A�@�G�T�|�����C�K�E�A�@�G�T�|�����C�K�E�A�@�G�T�|�����C�K�E�A�@�G�T�|�����C�K�E�C�@�G�T�|�����C�K�E�A�@�G�T�|�����C�K�E�A�@�G�T�|�����C�K�E�A�@�G�T�|�����C�K�E�A�@�G�T�|�����C�K�E�A�@�G�T�|�����C�K�E�A�@�G�T�|�����C�K�E�A�@�G�T�|�����C�K�E�A�@�G�T�|�����C�K�E�A�@�G�T�|�����C�K�E�C�@�G�T�|�����C�K�E�A�@�G�T�|�����C�K�E�A�@�G�T�|�����C�K�E�A�@�G�T�|�����C�K�E�A�@�G�T�|�����C�K�E�A�@�G�T�|�����C�K�E�A�@�G�T�|�����C�K�E�A�@�G�T�|�����C�K�E�A�@�G�T�|�����C�K�E�A�@�G�T�|�����C�K�E�C�@�G�T�|�����C�K�E�A�@�G�T�|�����C�K�E�A�@�G�T�|�����C�K�E�A�@�G�T�|�����C�K�E�A�@�G�T�|�����C�K�E�A�@�G�T�|�����C�K�E�A�@�G�T�|�����C�K�E�A�@�G�T�|�����C�K�E�A�@�G�T�|�����C�K�E�A�@�G�T�|�����C�K�E�C�@�G�T�|�����C�K�E�A�@�G�T�|�����C�K�E�A�@�G�T�|�����C�K�E�A�@�G�T�|�����C�K�E�A�@�G�T�|�����C�K�E�A�@�G�T�|�����C�K�E�A�@�G�T�|�����C�K�E�A�@�G�T�|�����C�K�E�A�@�G�T�|�����C�K�E�A�@�G�T�|�����C�K�E�C


%%%%%%%%%%%%%%%%%%%%%%%%%%%%%%%
%       �^��K�n 
%%%%%%%%%%%%%%%%%%%%%%%%%%%%%%%
%
\newpage
\thispagestyle{plain}  % �L header�A���b�B���L�Ҧ��U�|���B���L

% create an entry in table of contents for �^��K�n
\phantomsection % for hyperref to register this
\addcontentsline{toc}{chapter}{\nameEabstract}

% aligned to the center of the page
\begin{center}
% font size:
% \large (14pt) < \Large (18pt) < \LARGE (20pt) < \huge (24pt)< \Huge (24 pt)
% Set the line spacing to single for the names (to compress the lines)
\renewcommand{\baselinestretch}{1}   %��Z 1 ��
%\large % it needs a font size changing command to be effective
\large\eTitle\\  %�^���D��
\vspace{0.83cm}
% \makebox is a text box with specified width;
% option s: stretch
% use \makebox to make sure
% each text field occupies the same width
\makebox[2cm][l]{Student:}%
\makebox[5cm][l]{\myEname}%�ǥͭ^��m�W
\hfill%
%
\makebox[2cm][l]{Advisor:}%
\makebox[5cm][l]{\advisorEnameA}\\ %�б�A�^��m�W
%
% �P�_�O�_���@�P���ɪ��б� B
\ifx \advisorCnameB  \itsempty
\relax % �S�� B �б¡A�ҥH���������A���L����ť�
\else
%�@�P���ɪ��б�B
\makebox[2cm][l]{}%
\makebox[5cm][l]{}%%%%%
\hfill%
\makebox[2cm][l]{}%
\makebox[5cm][l]{\advisorEnameB}\\ %�б�B�^��m�W
\fi
%
% �P�_�O�_���@�P���ɪ��б� C
\ifx \advisorCnameC  \itsempty
\relax % �S�� C �б¡A�ҥH���������A���L����ť�
\else
%�@�P���ɪ��б�C
\makebox[2cm][l]{}%
\makebox[5cm][l]{}%%%%%
\hfill%
\makebox[2cm][l]{}%
\makebox[5cm][l]{\advisorEnameC}\\ %�б�C�^��m�W
\fi
%
\vspace{0.42cm}
Submitted to \fulldeptEname\\  %�^��t�ҥ��W
%
\ifx \collEname  \itsempty
\relax % �p�G�S���ǰ|�W (�^��)�A�h���������A���L����ť�
\else
% ���ǰ|�W (�^��)
\collEname\\% �ǰ|�W (�^��)
\fi
%
\univEname\\  %�^��զW
\vspace{0.83cm}
%\vfill
%
ABSTRACT\\
%\vspace{0.5cm}
\end{center}
% Resume the line spacing the desired setting
\renewcommand{\baselinestretch}{\mybaselinestretch}   %��_��]�w
%\large %it needs a font size changing command to be effective
% restore the font size to normal
\normalsize
%%%%%%%%%%%%%
The abstract written in English is placed here.  Please keep it in 250--500 words and limited in one page. A good abstract is analog to the shape of the hour glass, with thick top and bottom but a thin waist. The descriptions of your motivation and goal are as much as that of the effect and the application of your research result.  However, keep the description of the execution of your research in  minimal detail.  This enables the readers to quickly identify if this thesis answers their questions.  If it does, then the interested readers can continue to read the rest of the thesis. THE QUICK BROWN FOX JUMPS OVER THE LAZY DOG 01.  The quick brown fox jumps over the lazy dog 02.  The Quick Brown Fox Jumps Over the Lazy Dog 03.  The quick brown fox jumps over the lazy dog 04.  The Quick Brown Fox Jumps Over the Lazy Dog 05.  The quick brown fox jumps over the lazy dog 06.  The Quick Brown Fox Jumps Over the Lazy Dog 07.  The quick brown fox jumps over the lazy dog 08.  The Quick Brown Fox Jumps Over the Lazy Dog 09.  The quick brown fox jumps over the lazy dog 10.  The Quick Brown Fox Jumps Over the Lazy Dog 11.  The quick brown fox jumps over the lazy dog 12.  The Quick Brown Fox Jumps Over the Lazy Dog 13.  The quick brown fox jumps over the lazy dog 14.  The Quick Brown Fox Jumps Over the Lazy Dog 15.  The quick brown fox jumps over the lazy dog 16.  The Quick Brown Fox Jumps Over the Lazy Dog 17.  The quick brown fox jumps over the lazy dog 18.  The Quick Brown Fox Jumps Over the Lazy Dog 19.  The quick brown fox jumps over the lazy dog 20.  The quick brown fox jumps over the lazy dog 21.  The quick brown fox jumps over the lazy dog 22.


%%%%%%%%%%%%%%%%%%%%%%%%%%%%%%%
%       �x�� 
%%%%%%%%%%%%%%%%%%%%%%%%%%%%%%%
%
% Acknowledgment
\newpage
\chapter*{\protect\makebox[5cm][s]{\nameAckn}} %\makebox{} is fragile; need protect
\phantomsection % for hyperref to register this
\addcontentsline{toc}{chapter}{\nameAckn}
首先我要感謝我的指導老師。我也要感謝各位口試委員。我更要感謝我的父母。也要感謝實驗室的學長、學姊、學弟妹的鼓勵與幫助。 

%%%%%%%%%%%%%%%%%%%%%%%%%%%%%%%
%       �ؿ� 
%%%%%%%%%%%%%%%%%%%%%%%%%%%%%%%
%
% Table of contents
\newpage
\renewcommand{\contentsname}{\protect\makebox[5cm][s]{\nameToc}}
%\makebox{} is fragile; need protect
\phantomsection % for hyperref to register this
\addcontentsline{toc}{chapter}{\nameToc}
\tableofcontents

%%%%%%%%%%%%%%%%%%%%%%%%%%%%%%%
%       ���ؿ� 
%%%%%%%%%%%%%%%%%%%%%%%%%%%%%%%
%
% List of Tables
\newpage
\renewcommand{\listtablename}{\protect\makebox[5cm][s]{\nameLot}}
%\makebox{} is fragile; need protect
\phantomsection % for hyperref to register this
\addcontentsline{toc}{chapter}{\nameLot}
\listoftables

%%%%%%%%%%%%%%%%%%%%%%%%%%%%%%%
%       �ϥؿ� 
%%%%%%%%%%%%%%%%%%%%%%%%%%%%%%%
%
% List of Figures
\newpage
\renewcommand{\listfigurename}{\protect\makebox[5cm][s]{\nameTof}}
%\makebox{} is fragile; need protect
\phantomsection % for hyperref to register this
\addcontentsline{toc}{chapter}{\nameTof}
\listoffigures

%%%%%%%%%%%%%%%%%%%%%%%%%%%%%%%
%       �{���C���ؿ� 
%%%%%%%%%%%%%%%%%%%%%%%%%%%%%%%
%
% List of Listings
% �p�G�ݭn�{���C���ؿ��A�h�H�U�|��C�@�檺�歺�������Ÿ� % �R���H�Ѱ��ʦL
%\newpage
%\phantomsection % for hyperref to register this
%\addcontentsline{toc}{chapter}{\lstlistlistingname}
%\lstlistoflistings

%%%%%%%%%%%%%%%%%%%%%%%%%%%%%%%
%       �Ÿ����� 
%%%%%%%%%%%%%%%%%%%%%%%%%%%%%%%
%
% Symbol list
% define new environment, based on standard description environment
% adapted from p.60~64, <<The LaTeX Companion>>, 1994, ISBN 0-201-54199-8
\newcommand{\SymEntryLabel}[1]%
  {\makebox[3cm][l]{#1}}
%
\newenvironment{SymEntry}
   {\begin{list}{}%
       {\renewcommand{\makelabel}{\SymEntryLabel}%
        \setlength{\labelwidth}{3cm}%
        \setlength{\leftmargin}{\labelwidth}%
        }%
   }%
   {\end{list}}
%%
\newpage
\chapter*{\protect\makebox[5cm][s]{\nameSlist}} %\makebox{} is fragile; need protect
\phantomsection % for hyperref to register this
\addcontentsline{toc}{chapter}{\nameSlist}
%
% this file is encoded in utf-8
% v2.0 (Apr. 5, 2009)
%  各符號以 \item[] 包住,然後接著寫說明
% 如果符號是數學符號,應以數學模式$$表示,以取得正確的字體
% 如果符號本身帶有方括號,則此符號可以用大括號 {} 包住保護
\begin{SymEntry}

\item[OLED]
Organic Light Emitting Diode

\item[$E$]
energy

\item[$e$]
the absolute value of the electron charge, $1.60\times10^{-19}\,\text{C}$
 
\item[$\mathscr{E}$]
electric field strength (V/cm)

\item[{$A[i,j]$}]
the  element of the matrix $A$ at $i$-th row, $j$-th column\\
矩陣 $A$ 的第 $i$ 列,第 $j$ 行的元素

\end{SymEntry}


\newpage
%% �פ奻�魶�X�^�_�����ԧB�Ʀr�p���A�ñq�Y�_��
\pagenumbering{arabic}
%%%%%%%%%%%%%%%%%%%%%%%%%%%%%%%%
 

% 載入自訂巨集定義檔
% 這裡定義的巨集的使用說明,可以參考 example_macro_demo.pdf
% 預設載入。
% 如果不想要使用,把下一行行首加入百分號。
%% non-float version of table and figure environments
\makeatletter
\newenvironment{tablehere}
  {\def\@captype{table}}
  {}

\newenvironment{figurehere}
  {\def\@captype{figure}}
  {}
\makeatother


% some useful macros
%

% a single figure
%
% usage:
% \fig{width}
% {path/filename}
% {caption}
% {label}
\newcommand{\fig}[4]{
\begin{figure}[tbph]
{\centering \includegraphics[%
  width=#1,%
  keepaspectratio]{#2}
 \caption{#3}
 \label{#4}
 \par
}
\end{figure}
}

% a single figure, with additional text beneath the caption.  Usually the source of the figure.
%
% usage:
% \figt{width}
% {path/filename}
% {caption}
% {label}
% {additional text beneath the caption.  Usually the source of the figure.}
\newcommand{\figt}[5]{
\begin{figure}[tbph]
{\centering \includegraphics[%
  width=#1,%
  keepaspectratio]{#2}
 \caption{#3}
 \label{#4}
 #5\par
 }
\end{figure}
}

% single figure, additional text below the caption, spacing control between the figure and the caption
%
% usage:
% \figts{width}
% {path/filename}
% {caption}
% {label}
% {additional text beneath the caption.  Usually the source of the figure.}
% {place negative height, say, -4ex to reduce the gap between fig and caption}
\newcommand{\figts}[6]{
\begin{figure}[tbph]
{\centering \includegraphics[%
  width=#1,%
  keepaspectratio]{#2}
  \vspace{#6}\par
 \caption{#3}
 \label{#4}
 #5\par
 }
\end{figure}
}

% single figure, additional text below the caption, spacing control between the figure and the caption, trimming
%
% usage:
% \figtsr{width}
% {path/filename}
% {caption}
% {label}
% {additional text beneath the caption.  Usually the source of the figure.}
% {place negative height, say, -4ex to reduce the gap between fig and caption}
% {left bottom right top} in unit of bp (=1/72 in)
\newcommand{\figtsr}[7]{
\begin{figure}[tbph]
{\centering \includegraphics[%
  width=#1,%
  keepaspectratio,%
  trim=#7,clip=true]{#2}
  \vspace{#6}\par
 \caption{#3}
 \label{#4}
 #5\par
 }
\end{figure}
}

% multiple figures and captions in side-by-side arrangement;
%
% usage: (must be within figure environment)
% \mpfig{width}% in terms of \columnwidth
% {path/filename}
% {caption}
% {label}
\newcommand{\mpfig}[4]{
\begin{minipage}[b][1\totalheight]{#1\columnwidth}
\par\vspace{0pt}
{\centering \includegraphics[width=1\columnwidth,keepaspectratio]{#2}
\caption{#3}
\label{#4}\par}
\end{minipage}
}

% multiple figures and captions in side-by-side arrangement;
% each has its own descriptive text beneath the caption.
%
% usage: (must be within figure environment)
% \mpfigt{width}% in terms of \columnwidth
% {path/filename}
% {caption}
% {label}
% {additional text beneath the caption.  Usually the source of the figure.}
\newcommand{\mpfigt}[5]{
\begin{minipage}[b][1\totalheight]{#1\columnwidth}
{\centering \includegraphics[width=1\columnwidth,keepaspectratio]{#2}
\caption{#3}
\label{#4}
#5\par}
\end{minipage}
}

% multiple figures and captions in side-by-side arrangement, with spacing gap control;
%
% usage: (must be within figure environment)
% \mpfigts{width}% in terms of \columnwidth
% {path/filename}
% {caption}
% {label}
% {additional text beneath the caption.  Usually the source of the figure.}
% {place negative height, say, -4ex to reduce the gap between fig and caption}
\newcommand{\mpfigts}[6]{
\begin{minipage}[b][1\totalheight]{#1\columnwidth}
\par\vspace{0pt}
{\centering \includegraphics[width=1\columnwidth,keepaspectratio]{#2}
\vspace{#6}\par
\caption{#3}
\label{#4}
#5\par}
\end{minipage}
}

% multiple figures and captions in side-by-side arrangement, with spacing, w/ trim;
%
% usage: (must be within figure environment)
% \mpfigtra{width}% in terms of \columnwidth
% {path/filename}
% {caption}
% {label}
% {additional text beneath the caption.  Usually the source of the figure.}
% {place negative height, say, -4ex to reduce the gap between fig and caption}
% {left bottom right top} in terms of bp (=1/72 in)
\newcommand{\mpfigtsr}[7]{
\begin{minipage}[b][1\totalheight]{#1\columnwidth}
\par\vspace{0pt}
{\centering \includegraphics[width=1\columnwidth,keepaspectratio,%
 trim=#7,clip=true]{#2}
\vspace{#6}\par
\caption{#3}
\label{#4}
#5\par}
\end{minipage}
}

% multiple figures and captions in side-by-side arrangement, w/ spacing, w/ trim, w/ alignment;
%
% usage: (must be within figure environment)
% \mpfigtsra{width}% in terms of \columnwidth
% {path/filename}
% {caption}
% {label}
% {additional text beneath the caption.  Usually the source of the figure.}
% {place negative height, say, -4ex to reduce the gap between fig and caption}
% {left bottom right top}
% {alignment}
\newcommand{\mpfigtsra}[8]{
\begin{minipage}[#8][1\totalheight]{#1\columnwidth}
\par\vspace{0pt}
{\centering \includegraphics[width=1\columnwidth,keepaspectratio,%
 trim=#7,clip=true]{#2}
\vspace{#6}\par
\caption{#3}
\label{#4}
#5\par}
\end{minipage}
}

% multiple figures in side-by-side arrangement, but with a single caption
%
% usage: (must be within figure environment)
%     the caption capability is not included in this macro
%
% \mpfigabc{width}% in terms of \columnwidth
% {path/filename}
% {text beneath the figure.  Usually for the numbering of the subfigure}
\newcommand{\mpfigabc}[3]{
\begin{minipage}[b][1\totalheight]{#1\columnwidth}
\par\vspace{0pt}
{\centering \includegraphics[width=1\columnwidth,keepaspectratio]{#2}
#3\par}
\end{minipage}
}

% multiple figures in side-by-side arrangement, but with a single caption, w/spacing
%
% usage: (must be within figure environment)
%     the caption capability is not included in this macro
%
% \mpfigabcs{width}% in terms of \columnwidth
% {path/filename}
% {text beneath the figure.  Usually for the numbering of the subfigure}
% {place negative height, say, -4ex to reduce the gap between fig and caption}
\newcommand{\mpfigabcs}[4]{
\begin{minipage}[b][1\totalheight]{#1\columnwidth}
\par\vspace{0pt}
{\centering \includegraphics[width=1\columnwidth,keepaspectratio]{#2}\par
\vspace*{#4}#3\par}
\end{minipage}
}

% multiple figures in side-by-side arrangement, but with a single caption, w/spacing, w/trimming
%
% usage: (must be within figure environment)
%     the caption capability is not included in this macro
%
% \mpfigabcsr{width}% in terms of \columnwidth
% {path/filename}
% {text beneath the figure.  Usually for the numbering of the subfigure}
% {place negative height, say, -4ex to reduce the gap between fig and caption}
% {left bottom right top}
\newcommand{\mpfigabcsr}[5]{
\begin{minipage}[b][1\totalheight]{#1\columnwidth}
\par\vspace{0pt}
{\centering \includegraphics[width=1\columnwidth,keepaspectratio,%
 trim=#5,clip=true]{#2}\par
\vspace*{#4}#3\par}
\end{minipage}
}



% a single figure; FIGUREHERE version
%
% usage:
% \hfig{width}
% {path/filename}
% {caption}
% {label}
\newcommand{\hfig}[4]{
\begin{figurehere}
{\centering \includegraphics[%
  width=#1,%
  keepaspectratio]{#2}
\caption{#3}
\label{#4}
\par
}
\end{figurehere}
}

% a single figure, with additional text beneath the caption.  Usually the source of the figure. FIGUREHERE version
%
% usage:
% \hfigt{width}
% {path/filename}
% {caption}
% {label}
% {additional text beneath the caption.  Usually the source of the figure.}
\newcommand{\hfigt}[5]{
\begin{figurehere}
{\centering \includegraphics[%
  width=#1,%
  keepaspectratio]{#2}%
\caption{#3}
\label{#4}
#5\par}
\end{figurehere}
}

% controlling the gap between fig and caption
% FIGUREHERE version
%
% usage:
% \hfigts{width}
% {path/filename}
% {caption}
% {label}
% {additional text beneath the caption.  Usually the source of the figure.}
% {place negative height, say, -4ex to reduce the gap between fig and caption}
\newcommand{\hfigts}[6]{
\begin{figurehere}
{\centering \includegraphics[%
  width=#1,%
  keepaspectratio]{#2}
\vspace{#6}\par
\caption{#3}
\label{#4}
#5\par}
\end{figurehere}
}

% single figure, additional text below the caption spacing control between the figure and the caption, trimming
% FIGUREHERE version
%
% usage:
% \hfigtsr{width}
% {path/filename}
% {caption}
% {label}
% {additional text beneath the caption.  Usually the source of the figure.}
% {place negative height, say, -4ex to reduce the gap between fig and caption}
% {left bottom right top} in unit of bp (= 1/72 in)
\newcommand{\hfigtsr}[7]{
\begin{figurehere}
{\centering \includegraphics[%
  width=#1,%
  keepaspectratio,%
  trim=#7,clip=true]{#2}
\vspace{#6}\par
\caption{#3}
\label{#4}
#5\par
}
\end{figurehere}
}



% main body 論文主體。建議以「章」為檔案分割的依據。
% 在下列各行,建議了
%   intro.tex, experiment.tex, theory.tex, calculation.tex, summary.tex
% 做為這幾個「章」的檔案名稱
% 實際命名方式可以隨你意
% 在撰寫各章草稿時,可以把其他章節關掉 (行首加百分號)
%
%
% this file is encoded in big5
% v2.03 (Feb. 26, 2013)

% �{�ɩw�q�F fmpage: �@�ӥ[�ت��i�ܰ� framed minipage
% http://brunoj.wordpress.com/2009/10/08/latex-the-framed-minipage/
\newsavebox{\fmbox}
\newenvironment{fmpage}[1]
{\begin{lrbox}{\fmbox}\begin{minipage}{#1}}
{\end{minipage}\end{lrbox}\fbox{\usebox{\fmbox}}}

\chapter{����}
\label{sec:intro}

�o�@���ت��O�@�ǰ򥻥\�઺�Ϊk�C�̦n�O�P��l�ɮ� \verb+example_body.tex+ �@�_��ӵ۬ݡC���`���D�ᤧ�Ĥ@�q���q�����ݭn���Y�C�o�۰ʥѮ榡�ɴx���C

�o�O�ĤG�q�C�ĤG�q���}�Y�|���Y�C�o�]�O�۰ʥѮ榡�ɴx���C�n�p����q�H�u�n�s����J��ӡuEnter (return)�v�A�N�N�����q�C

��Z�̥� \verb+\chapter{}+ �N���Y�@�����}�l�A�H�γ��W�C�C���̷|���X�Ӹ`�A�H \verb+\section+\{�`�W\} ���O�N���s���@�`���}�l�C\verb+\subsection+\{�p�`�W\} ���O�N���s���p�`���}�l�C

%%%%%%%%
\section{\mbox{\LaTeX\ �򥻥ܽd}}
\label{sec:latex_basic}

�Ҧ��� \LaTeX\ �B�z�s�����a��A�p���B�`�B�p�`�B�ϡB���B��{���B���C�A���i�H�b��Z�̤ޥΨ��ӽs���A�u�n�b���B�`�B�p�`�B�ϡB���B��{���B���C���ͪ��a�赹���@�Ӽ��� \verb+\label+\{�W�S�^��r��\}�A�M��b��Z��L�a��n�ޥΥ����s�����a��A�H \verb+\ref+\{�ҹ��������Ҧr��\} �Y�i�C

\LaTeX\ ���O�ѭ˱׽u \textbackslash\ �_�Y�A�ʤ����h�N�����Ѥ�r���_�Y�C��L�٦��X�ӯS�O�r�����ઽ���b��Z�̨ϥΡC�p�G�n�Ψ�A�Ьݦb�� \pageref{tb:special_chars} ����~\ref{tb:special_chars} ���u��Z�W�ϥΡv�Ϊ̡u\LaTeX\ �����N���O�v�o����A�ܤ@�ϥΡC�N�b�e�@�y�A�ڭ̥ΤF \verb+\ref{}+ �N������s���A�H�Υ��Ҧb�����X \verb+\pageref{}+�A�o���i�Ѩt�άd�߫�۰ʶ�J�C���K���O�A�e�����y�X�{������s���P���X�A�b PDF �s�����̥i�H�����ηƹ������I���A�i�H���D�쨺�Ӫ���B�������Ҧb�C�������\��]�b�פ�@�}�l���ؿ��B���ؿ��B�ϥؿ����̥i�H�ϥΡC�s���������A�]�����פ�U���`���C���A�i�H�����I��\Ū�C�t�~�A�ѩ��~\ref{tb:special_chars} �һݪ��������j�A�t�Φb�ƪ��ɷ|��������u�}�B�v��A�����a��ƪ��C�p�G���Ʊ楦�B�ʡA�h�n�ϥΡu�T�w�O���v�� tablehere ���O (�۳Х���)�C

%%%%%%%%%%%%%%
%%%%%%%%%%%%%% LaTeX special chars % 
\begin{table}[htbp] %t: top; b: bottom; p: in a page alone; h: here
\caption{\label{tb:special_chars}% 
\mbox{�L�k�����b \LaTeX\ ��Z�̨ϥΪ��Ÿ��r��}
}% 
\begin{center} \begin{tabular}{llll}% four left-aligned col, no vertical boarders
\hline 
�Ÿ�&
�@��&
��Z�W�ϥ�&
\LaTeX\ �����N���O\tabularnewline
\hline
\textbackslash &
�U�ƪ��R�O&
\verb+$\backslash$+&
\verb+\textbackslash+\tabularnewline
%
\%&
����&
\verb+\%+&
NA\tabularnewline
%
\#&
�w�q����&
\verb+\#+&
NA\tabularnewline
%
\~{}&
���_��ť�&
\verb+\~{}+&
\verb+\textasciitilde+\tabularnewline
%
\$&
�i�J (���}) �ƾǼҦ�&
\verb+\$+&
\verb+\textdollar+\tabularnewline
%
\_{}&
�ƾǼҦ����ͤU�Цr&
\verb+\_{}+&
\verb+\textunderscore+\tabularnewline
%
\^{}&
�ƾǼҦ����ͤW�Цr&
\verb+\^{}+&
\verb+\textasciicircum+\tabularnewline
%
\{&
�ХܩR�O���@�νd��&
\verb+\{+&
\verb+\textbraceleft+\tabularnewline
%
\}&
�ХܩR�O���@�νd��&
\verb+\}+&
\verb+\textbraceright+\tabularnewline
%
$<$&
�ƾǼҦ������p��Ÿ�&
\verb+$<$+&
\verb+\textless+\tabularnewline
%
$>$&
�ƾǼҦ������j��Ÿ�&
\verb+$>$+&
\verb+\textgreater+\tabularnewline
%
$|$&
�ƾǼҦ���������ȲŸ�&
\verb+$|$+&
\verb+\textbar+\tabularnewline
%
\&&
���椤�����j�Ÿ�&
\verb+\&+&
NA\tabularnewline
%
\hline
\end{tabular}\\
�����^�ۤ��m\cite{lee_latex123}
\end{center}
\end{table}
%%%%%%%%%%%%%%

\section{���J����}
\label{sec:table}

�b�� \pageref{tb:special_chars} ������~\ref{tb:special_chars} �b��Z�̪���l�X�ݨӦ��I�����C��򥻬[�c�O�o�˪��G
	\begin{center}
%	\begin{minipage}[t][1\totalheight]{0.3\columnwidth}
	\begin{fmpage}{0.3\columnwidth}
		�B�ʪO��\\
		\verb+\begin{table}[htbp]+\\
		\verb+\caption+\{������D��r\}\\
		���氩�[���O tabular\\
		\dots\\
		\verb+\end{table}+
%	\end{minipage}
	\end{fmpage}
	%
 	\hspace{3em} %�P�T�� M �r���P�e���ť�
	%
%	\begin{minipage}[t][1\totalheight]{0.5\columnwidth}
	\begin{fmpage}{0.5\columnwidth}
		�T�w�O�� (�D���إ\��A�w�q��۳Х���)\\
		\verb+\begin{tablehere}+\\
		\verb+\caption+\{������D��r\}\\
		���氩�[���O tabular\\
		\dots\\
		\verb+\end{tablehere}+
%	\end{minipage}
	\end{fmpage}
	\end{center}
���檺���D��r���b����W��A��ƪ��X�B�h���O�����U��C (�����j�ǽפ�W�d)

�p��غc���氩�[�H��~\ref{tb:income_2003} �O�ܽd²���� $3\times3$ ������C�ҥΤF tabular ���ҫ��O�p�U�A�䤤 \verb+{|c|r|r|}+ �H�^��r���P�ɪ����F���ơB������覡 (��  l (�p�g L)�B�� c�B�k r)�B�H�Ϋ������j�u $|$�G
	\begin{center}
%	\begin{minipage}[t][1\totalheight]{0.5\columnwidth}
	\begin{fmpage}{0.5\columnwidth}
		\verb+\begin{tabular}{|c|r|r|}+\\
		���椺�e \ldots\\
		\verb+\end{tabular}+
%	\end{minipage}
	\end{fmpage}
	\end{center}
%
\verb+{|c|r|r|}+ ���T�ӭ^��r�� c, r, r�A���ܦ��T��A���F�Ĥ@��O��������H�~�A�ĤG�B�T��O�V�k����A�P�ɦ��������j�u�b���k�ⰼ�H����P�椧���C���椺�e�A�H�@�C�@�C��J�A�C�������H�_��Ÿ� \verb+\\+ �Ϊ� \verb+\tabularnewline+ ���ܡF�C�@�C�̡A��P�椧���H \&\ �Ÿ��j�}�C�p�G�ݭn�������j�u�A�h�H \verb+\hline+ �N���C

�b�M�~���y�B���Z���榡�A����u�ϥξ�u�A�p��~\ref{tb:income_2003_v2}�C�����������s�@��k�A�i�H�ѦҤ��m\cite{url_latex_wikibook_table}�C���檺���O���T�����A�O�H���Ἲ�áC�ҩ��i�H�ϥ� \texttt{excel2latex} �u�㪽���q \texttt{Excel} �պ�����γn���ഫ�� \LaTeX\ ����X�K�J��Z�ϥ�\cite{url_excel2latex}�A�Ҧp��~\ref{tab:excel2latex}�C�Ϊ̪����b�����W�s�@�����ұo������X�K�J��Z�ϥ�\cite{url_table_editor}�C

\begin{table}[htbp]
\caption{\label{tb:income_2003}Gross income of the first two quarters of
year 2003}
\begin{center} \begin{tabular}{|c|r|r|}
\hline 
&
Restaurant&
Store\tabularnewline
\hline
\hline 
Q1&
\$123,000&
\$75,000\tabularnewline
\hline 
Q2&
\$45,000&
\$131,000\tabularnewline
\hline
\end{tabular}\\
(This table is made up for demonstration purpose.)
\end{center}
\end{table}%
%

\begin{table}[htbp]
\caption{\label{tb:income_2003_v2}Gross income of the first two quarters of
year 2003. Version 2}
\begin{center} \begin{tabular}{crr}
\hline 
&
Restaurant&
Store\tabularnewline
\hline
Q1&
\$123,000&
\$75,000\tabularnewline
 Q2&
\$45,000&
\$131,000\tabularnewline
\hline
\end{tabular}\\
(This table is made up for demonstration purpose.)
\end{center}
\end{table}

% Table generated by Excel2LaTeX from sheet
\begin{table}[htbp]
  \centering
  \caption{�� excel2latex ��U���ͪ�����}
  \vspace*{1ex}
    \begin{tabular}{|r|c|c|c|c|c|c|}
    \hline
          & \multicolumn{2}{c|}{Sample 1} & \multicolumn{2}{c|}{Sample 2} & \multicolumn{2}{c|}{Sample 3} \bigstrut\\
    \hline
          & $V_L$    & $V_H$    &  $V_L$    & $V_H$     &  $V_L$    & $V_H$  \bigstrut\\
    \hline
    Condition 1 & 1.5   & 2.6   & 1.3   & 2.4   & 1.6   & 2.6 \bigstrut\\
    \hline
    Condition 2 & 1.4   & 2.6   & 1.2   & 2.4   & 1.5   & 2.6 \bigstrut\\
    \hline
    Condition 3 & 1.6   & 2.6   & 1.4   & 2.5   & 1.7   & 2.7 \bigstrut\\
    \hline
    \end{tabular}%
  \label{tab:excel2latex}%
\end{table}
\clearpage
%%%%%%%%%%%%%%
\section{���J����}
\label{sec:fig}

�פ�̤֤��F�n�m��Ϥ��A�Ҧp����ƾ��ͶչϡA����]�ƥܷN�ϡC�P�@���ѳB�z�� MS WORD ���P���O�A�o�ǹϥ����U�ۿW�߬��@���ɮסC�b�o�@�`�ܽd�F�@�ǥ������O�C�o�ǥ������O�O���F��K�B�z���ϦӼg�A�ä��O�зǤ��ت��C�w�q�� \verb+npc_macros_20120912.tex+ �̡A�i�H�b��Z���A�����a��[�J�w�q�ɡA��k�p�U�G\\
\verb+%% non-float version of table and figure environments
\makeatletter
\newenvironment{tablehere}
  {\def\@captype{table}}
  {}

\newenvironment{figurehere}
  {\def\@captype{figure}}
  {}
\makeatother


% some useful macros
%

% a single figure
%
% usage:
% \fig{width}
% {path/filename}
% {caption}
% {label}
\newcommand{\fig}[4]{
\begin{figure}[tbph]
{\centering \includegraphics[%
  width=#1,%
  keepaspectratio]{#2}
 \caption{#3}
 \label{#4}
 \par
}
\end{figure}
}

% a single figure, with additional text beneath the caption.  Usually the source of the figure.
%
% usage:
% \figt{width}
% {path/filename}
% {caption}
% {label}
% {additional text beneath the caption.  Usually the source of the figure.}
\newcommand{\figt}[5]{
\begin{figure}[tbph]
{\centering \includegraphics[%
  width=#1,%
  keepaspectratio]{#2}
 \caption{#3}
 \label{#4}
 #5\par
 }
\end{figure}
}

% single figure, additional text below the caption, spacing control between the figure and the caption
%
% usage:
% \figts{width}
% {path/filename}
% {caption}
% {label}
% {additional text beneath the caption.  Usually the source of the figure.}
% {place negative height, say, -4ex to reduce the gap between fig and caption}
\newcommand{\figts}[6]{
\begin{figure}[tbph]
{\centering \includegraphics[%
  width=#1,%
  keepaspectratio]{#2}
  \vspace{#6}\par
 \caption{#3}
 \label{#4}
 #5\par
 }
\end{figure}
}

% single figure, additional text below the caption, spacing control between the figure and the caption, trimming
%
% usage:
% \figtsr{width}
% {path/filename}
% {caption}
% {label}
% {additional text beneath the caption.  Usually the source of the figure.}
% {place negative height, say, -4ex to reduce the gap between fig and caption}
% {left bottom right top} in unit of bp (=1/72 in)
\newcommand{\figtsr}[7]{
\begin{figure}[tbph]
{\centering \includegraphics[%
  width=#1,%
  keepaspectratio,%
  trim=#7,clip=true]{#2}
  \vspace{#6}\par
 \caption{#3}
 \label{#4}
 #5\par
 }
\end{figure}
}

% multiple figures and captions in side-by-side arrangement;
%
% usage: (must be within figure environment)
% \mpfig{width}% in terms of \columnwidth
% {path/filename}
% {caption}
% {label}
\newcommand{\mpfig}[4]{
\begin{minipage}[b][1\totalheight]{#1\columnwidth}
\par\vspace{0pt}
{\centering \includegraphics[width=1\columnwidth,keepaspectratio]{#2}
\caption{#3}
\label{#4}\par}
\end{minipage}
}

% multiple figures and captions in side-by-side arrangement;
% each has its own descriptive text beneath the caption.
%
% usage: (must be within figure environment)
% \mpfigt{width}% in terms of \columnwidth
% {path/filename}
% {caption}
% {label}
% {additional text beneath the caption.  Usually the source of the figure.}
\newcommand{\mpfigt}[5]{
\begin{minipage}[b][1\totalheight]{#1\columnwidth}
{\centering \includegraphics[width=1\columnwidth,keepaspectratio]{#2}
\caption{#3}
\label{#4}
#5\par}
\end{minipage}
}

% multiple figures and captions in side-by-side arrangement, with spacing gap control;
%
% usage: (must be within figure environment)
% \mpfigts{width}% in terms of \columnwidth
% {path/filename}
% {caption}
% {label}
% {additional text beneath the caption.  Usually the source of the figure.}
% {place negative height, say, -4ex to reduce the gap between fig and caption}
\newcommand{\mpfigts}[6]{
\begin{minipage}[b][1\totalheight]{#1\columnwidth}
\par\vspace{0pt}
{\centering \includegraphics[width=1\columnwidth,keepaspectratio]{#2}
\vspace{#6}\par
\caption{#3}
\label{#4}
#5\par}
\end{minipage}
}

% multiple figures and captions in side-by-side arrangement, with spacing, w/ trim;
%
% usage: (must be within figure environment)
% \mpfigtra{width}% in terms of \columnwidth
% {path/filename}
% {caption}
% {label}
% {additional text beneath the caption.  Usually the source of the figure.}
% {place negative height, say, -4ex to reduce the gap between fig and caption}
% {left bottom right top} in terms of bp (=1/72 in)
\newcommand{\mpfigtsr}[7]{
\begin{minipage}[b][1\totalheight]{#1\columnwidth}
\par\vspace{0pt}
{\centering \includegraphics[width=1\columnwidth,keepaspectratio,%
 trim=#7,clip=true]{#2}
\vspace{#6}\par
\caption{#3}
\label{#4}
#5\par}
\end{minipage}
}

% multiple figures and captions in side-by-side arrangement, w/ spacing, w/ trim, w/ alignment;
%
% usage: (must be within figure environment)
% \mpfigtsra{width}% in terms of \columnwidth
% {path/filename}
% {caption}
% {label}
% {additional text beneath the caption.  Usually the source of the figure.}
% {place negative height, say, -4ex to reduce the gap between fig and caption}
% {left bottom right top}
% {alignment}
\newcommand{\mpfigtsra}[8]{
\begin{minipage}[#8][1\totalheight]{#1\columnwidth}
\par\vspace{0pt}
{\centering \includegraphics[width=1\columnwidth,keepaspectratio,%
 trim=#7,clip=true]{#2}
\vspace{#6}\par
\caption{#3}
\label{#4}
#5\par}
\end{minipage}
}

% multiple figures in side-by-side arrangement, but with a single caption
%
% usage: (must be within figure environment)
%     the caption capability is not included in this macro
%
% \mpfigabc{width}% in terms of \columnwidth
% {path/filename}
% {text beneath the figure.  Usually for the numbering of the subfigure}
\newcommand{\mpfigabc}[3]{
\begin{minipage}[b][1\totalheight]{#1\columnwidth}
\par\vspace{0pt}
{\centering \includegraphics[width=1\columnwidth,keepaspectratio]{#2}
#3\par}
\end{minipage}
}

% multiple figures in side-by-side arrangement, but with a single caption, w/spacing
%
% usage: (must be within figure environment)
%     the caption capability is not included in this macro
%
% \mpfigabcs{width}% in terms of \columnwidth
% {path/filename}
% {text beneath the figure.  Usually for the numbering of the subfigure}
% {place negative height, say, -4ex to reduce the gap between fig and caption}
\newcommand{\mpfigabcs}[4]{
\begin{minipage}[b][1\totalheight]{#1\columnwidth}
\par\vspace{0pt}
{\centering \includegraphics[width=1\columnwidth,keepaspectratio]{#2}\par
\vspace*{#4}#3\par}
\end{minipage}
}

% multiple figures in side-by-side arrangement, but with a single caption, w/spacing, w/trimming
%
% usage: (must be within figure environment)
%     the caption capability is not included in this macro
%
% \mpfigabcsr{width}% in terms of \columnwidth
% {path/filename}
% {text beneath the figure.  Usually for the numbering of the subfigure}
% {place negative height, say, -4ex to reduce the gap between fig and caption}
% {left bottom right top}
\newcommand{\mpfigabcsr}[5]{
\begin{minipage}[b][1\totalheight]{#1\columnwidth}
\par\vspace{0pt}
{\centering \includegraphics[width=1\columnwidth,keepaspectratio,%
 trim=#5,clip=true]{#2}\par
\vspace*{#4}#3\par}
\end{minipage}
}



% a single figure; FIGUREHERE version
%
% usage:
% \hfig{width}
% {path/filename}
% {caption}
% {label}
\newcommand{\hfig}[4]{
\begin{figurehere}
{\centering \includegraphics[%
  width=#1,%
  keepaspectratio]{#2}
\caption{#3}
\label{#4}
\par
}
\end{figurehere}
}

% a single figure, with additional text beneath the caption.  Usually the source of the figure. FIGUREHERE version
%
% usage:
% \hfigt{width}
% {path/filename}
% {caption}
% {label}
% {additional text beneath the caption.  Usually the source of the figure.}
\newcommand{\hfigt}[5]{
\begin{figurehere}
{\centering \includegraphics[%
  width=#1,%
  keepaspectratio]{#2}%
\caption{#3}
\label{#4}
#5\par}
\end{figurehere}
}

% controlling the gap between fig and caption
% FIGUREHERE version
%
% usage:
% \hfigts{width}
% {path/filename}
% {caption}
% {label}
% {additional text beneath the caption.  Usually the source of the figure.}
% {place negative height, say, -4ex to reduce the gap between fig and caption}
\newcommand{\hfigts}[6]{
\begin{figurehere}
{\centering \includegraphics[%
  width=#1,%
  keepaspectratio]{#2}
\vspace{#6}\par
\caption{#3}
\label{#4}
#5\par}
\end{figurehere}
}

% single figure, additional text below the caption spacing control between the figure and the caption, trimming
% FIGUREHERE version
%
% usage:
% \hfigtsr{width}
% {path/filename}
% {caption}
% {label}
% {additional text beneath the caption.  Usually the source of the figure.}
% {place negative height, say, -4ex to reduce the gap between fig and caption}
% {left bottom right top} in unit of bp (= 1/72 in)
\newcommand{\hfigtsr}[7]{
\begin{figurehere}
{\centering \includegraphics[%
  width=#1,%
  keepaspectratio,%
  trim=#7,clip=true]{#2}
\vspace{#6}\par
\caption{#3}
\label{#4}
#5\par
}
\end{figurehere}
}

+\\
�w�]�O���[�J�A�b  \verb+my_chapters.tex+ ���C

���U�ܽd�F�o�Ǧ۳ЩR�O���Ϊk�C���ɬO�b \texttt{figs} �l�ؿ��̡A�W�r�O \verb+example_fig.png+�A�b�H�U�����O���A�Ϫ��ɦW�����]�t \verb+.png+ �o�Ӱ��ɦW�C (�o�̤]�ܽd�F�۰ʽs�������C�M��A�s�� \texttt{enumerate} ���ҫ��O�A�C�@���� \verb+\item+ �_�Y)


\begin{enumerate}
%%%%%%%%%%%%%%%%%%%%%%%%
\item \verb+\fig+ �N����i�Ϫ����O (single figure)�A �w���@�i�ϩ�B�ʪO�� (�аѦҹ�~\ref{fig:yzu_logo_1})�C
�ϼe�ת����i�H�ۦ���w�A�Ҧp���Ҫ� 2~cm�A�H�α`�Ϊ������i���r�ϼe�� \verb+\columnwidth+�C�O�����e��ҡA���������ܡC���Ϫ����D���b�Ϫ��U��C�p�G�n�Ρu�T�w�O���v�A�h�� \verb+\fig+ �אּ \verb+\hfig+ �N��O figure here�C

{\centering\begin{lstlisting}[caption={fig �ϥνd�ҽX},
label=lst:fig,
numbers=left,
firstnumber=1,
frame=ltrb, % single lines for left, top, right, bottom; LTRB for double lines 
%escapeinside={$$}, %�p�n�b�C������ܯS���r��/�ƪ��ĪG�A�n��Ӥ�r��� $$ �]���� (�A�X C �{���X)(��w�]�� <>)
]
\fig{2cm} %width (specify the unit: e.g., 0.75\columnwidth)
{figs/example_fig} %path/filename (no space)
{<�ϥ� fig �i�ܤ�������>} %caption
{fig:yzu_logo_1} %label
\end{lstlisting}\par}

% 
%
%%%%%%%%%%%%
%\hfig %for non-float
%
\fig{2cm} %width; specify the unit: e.g., 0.75\columnwidth
{figs/example_fig} %path/filename (no space)
{�ϥ� fig �i�ܤ�������} %caption
{fig:yzu_logo_1} %label
%
%%%%%%%%%%%%

%%%%%%%%%%%%%%%%%%%%%%%%
\item \verb+\figt+ �N�� single figure with additional text �P \verb+\fig+ �ۦP�A���O���D���U�i�H�t����r (\uline{t}ext) �����Ϫ��X�B�����C(�аѦҹ�~\ref{fig:yzu_logo_2}) �p�G�n�Ρu�T�w�O���v�A�h�� \verb+\figt+ �אּ \verb+\hfigt+ �N��O figure here ���� \verb+figt+�C

{\centering\begin{lstlisting}[caption={figt �ϥνd�ҽX},
label=lst:figt,
numbers=left,
firstnumber=1,
frame=ltrb, % single lines for left, top, right, bottom; LTRB for double lines 
%escapeinside={$$}, %�p�n�b�C������ܯS���r��/�ƪ��ĪG�A�n��Ӥ�r��� $$ �]���� (�A�X C �{���X)(��w�]�� <>)
]
\figt{2cm} %width; specify the unit: e.g., 0.75\columnwidth
{figs/example_fig} %path/filename (no space)
{<�ϥ� figt �i�ܦ������X�B����������>} %caption
{fig:yzu_logo_2} %label
{<�^���ۤ����j�Ǻ���>} %additional text beneath the caption
\end{lstlisting}\par}

%%%%%%%%%%%%
%\hfigt % for non-float
%
\figt{2cm} %width (specify the unit: e.g., 0.75\columnwidth)
{figs/example_fig} %path/filename (no space)
{�ϥ� figt �i�ܦ������X�B����������} %caption
{fig:yzu_logo_2} %label
{�^���ۤ����j�Ǻ���} %additional text beneath the caption
%
%%%%%%%%%%%%

%%%%%%%%%%%%%%%%%%%%%%%%
\item \verb+\figts+ �w���@�i�ϡA�i�t�X�B��r (\uline{t}ext)�A�٥i�H����ϻP���D��r�������Ż� (\uline{s}pacing)�C�s���Ѽơu���Z����v�i�H�վ�ϩ��U�P���D������r���������Z�C�q�`�O�]�����Z�Ӥj�ӧƱ��Ԫ�ϻP���D��r�C�Q�Ԫ�Z���A�h�έt�����סA�Ҧp $-1$cm�A$-4$ex (�H�^��r�� x �����׬����)�C�p�G�n�H��l���Z�e�{�A�h�� 0cm, 0ex�C�аѦҹ�~\ref{fig:yzu_logo_3}�C�p�G�n�Ρu�T�w�O���v�A�h�� \verb+\figts+ �אּ \verb+\hfigts+ �N��O figure here ���� \verb+figts+�C


{\centering\begin{lstlisting}[caption={figts �ϥνd�ҽX},
label=lst:figts,
numbers=left,
firstnumber=1,
frame=ltrb, % single lines for left, top, right, bottom; LTRB for double lines 
%escapeinside={$$}, %�p�n�b�C������ܯS���r��/�ƪ��ĪG�A�n��Ӥ�r��� $$ �]���� (�A�X C �{���X)(��w�]�� <>)
]
\figts{2cm} %width; specify the unit: e.g., 0.75\columnwidth
{figs/example_fig} %path/filename (no space)
{<�ϥ� figts �e�{���񪺼��D��r>} %caption
{fig:yzu_logo_3} %label
{<�^���ۤ����j�Ǻ���>} %additional text beneath the caption
{-2.5ex} %place negative height, say, -4ex to reduce the gap between fig and caption
\end{lstlisting}\par}

%%%%%%%%%%%%
%\hfigtr % for non-float
%
\figts{2cm} %width (specify the unit: e.g., 0.75\columnwidth)
{figs/example_fig} %path/filename (no space)
{�ϥ� figts �e�{���񪺼��D��r} %caption
{fig:yzu_logo_3} %label
{����^���ۤ����j�Ǻ���} %additional text beneath the caption
{-2.5ex} %place negative height, say, -4ex to reduce the gap between fig and caption
%
%%%%%%%%%%%%

%%%%%%%%%%%%%%%%%%%%%%%%
\item \verb+\figtsr+ ���F�e�z���ӷ���r (\uline{t}ext), �i�H����ϻP���D��r�������Ż� (\uline{s}pacing)�A�٦����� (t\uline{r}im) �\��C
�Ϫ��|�P���ťեi�H�����C(�аѦҹ�~\ref{fig:yzu_logo_3_2}) �s�[�J���Ѽƪ��Ʀr�N�q�p�U�G���B�U�B�k�B�W�A�n���������סA�H bp �����A1~bp $= \frac{1}{72}$~inch�C
�аѦҹ�~\ref{fig:yzu_logo_3_2}�C
���r�ѼƪŵۡB���Z�W�q�]�� 0cm�B�����q 0 0 0 0�A�h�o���O�]�i�H�����\����֪� \verb+\fig+, \verb+\figt+, \verb+\figts+ �ӥΡC �p�G�n�Ρu�T�w�O���v�A�h�� \verb+\figtsr+ �אּ \verb+\hfigtsr+ �N��O figure here ���� \verb+figtsr+�C

{\centering\begin{lstlisting}[caption={figtsr �ϥνd�ҽX},
label=lst:figtsr,
numbers=left,
firstnumber=1,
frame=ltrb, % single lines for left, top, right, bottom; LTRB for double lines 
%escapeinside={$$}, %�p�n�b�C������ܯS���r��/�ƪ��ĪG�A�n��Ӥ�r��� $$ �]���� (�A�X C �{���X)(��w�]�� <>)
]
\figtsr{2cm} %width; specify the unit
{figs/example_fig} %path/filename (no space)
{<�ϥ� figtrs �i�ܤ��Ϋ᪺�����������P���D��r��Z���p>} %caption
{fig:yzu_logo_3_2} %label
{<����^���ۤ����j�Ǻ���>} %additional text beneath the caption
{-2.5ex}% place negative height, say, -4ex to reduce the gap between fig and caption
{20 20 20 20} %left bottom right top, in units of bp=1/72 in
\end{lstlisting}\par}

%%%%%%%%%%%%
%\hfigtr % for non-float
%
\figtsr{2cm} %width; specify the unit
{figs/example_fig} %path/filename (no space)
{�ϥ� figtsr �i�ܤ��Ϋ᪺���������B�P���D��r��Z���p} %caption
{fig:yzu_logo_3_2} %label
{����^���ۤ����j�Ǻ���} %additional text beneath the caption
{-2.5ex}% place negative height, say, -4ex to reduce the gap between fig and caption
{20 20 20 20} %left bottom right top, in units of bp=1/72 in
%
%%%%%%%%%%%%

%%%%%%%%%%%%%%%%%%%%%%%%
\item \verb+\mpfig+ �h�i�Ϩñ� (multiple figures)�A�C�i���U�ۿW�ߪ��ϼ��D (caption) (�аѦҹ�~\ref{fig:yzu_logo_4} �P \ref{fig:yzu_logo_5})�C
�ϼe�ת����w���w�A�H�����i���r�ϼe�׬� 1�A�Ҧp���Ҫ� 0.2 �N�����ϼe�׬O�����e�� 0.2�A�Y 20\% �����i���r�ϼe�C�p�G�n�ñƧ�h���ϡA�u�n�b \verb+\end{figure}+ ���e�A��J��h�� \verb+\hfill+ �P \verb+\mpfig+ �Y�i�C���M�A�U�ӹϪ��e���`�M���n�W�L 1�C�p�G�n�Ρu�T�w�O���v�A�h�� \verb+\begin{figure}[tbph]+ �P \verb+\end{figure}+ �אּ \verb+\begin{figurehere}+ �P \verb+\end{figurehere}+�C

{\centering\begin{lstlisting}[caption={mpfig �ϥνd�ҽX},
label=lst:mpfig,
numbers=left,
firstnumber=1,
frame=ltrb, % single lines for left, top, right, bottom; LTRB for double lines 
%escapeinside={$$}, %�p�n�b�C������ܯS���r��/�ƪ��ĪG�A�n��Ӥ�r��� $$ �]���� (�A�X C �{���X)(��w�]�� <>)
]
\begin{figure}[tbph]
%
\mpfig{0.25} %width in terms of \columnwidth
{figs/example_fig} %path/filename (no space)
{<�ϥ� mpfig �i�ܤ������� 1>} %caption
{fig:yzu_logo_4} %label
%
\hfill %flexible gap in-between
%
\mpfig{0.25} %width in terms of \columnwidth
{figs/example_fig} %path/filename (no space)
{<�ϥ� mpfig �i�ܤ������� 2>} %caption
{fig:yzu_logo_5} %label
%
\end{figure}
\end{lstlisting}\par}

%\vspace*{5cm}

%%%%%%%%%%
% \begin{figurehere} % for non-float
%
\begin{figure}[tbph]
%
\mpfig{0.25} %width in terms of \columnwidth
{figs/example_fig} %path/filename (no space)
{�ϥ� mpfig �i�ܤ������� 1} %caption
{fig:yzu_logo_4} %label
%
\hfill
%
\mpfig{0.25} %width in terms of \columnwidth
{figs/example_fig} %path/filename (no space)
{�ϥ� mpfig �i�ܤ������� 2} %caption
{fig:yzu_logo_5} %label
%
\end{figure}
% \end{figurehere} % for non-float
%%%%%%%%%%%%


%%%%%%%%%%%%%%%%%%%%%%%%
\item \verb+\mpfigt+ �P�e�z \verb+\mpfig+ �ۦP�A���O�U�ۼ��D���U�t����r (text) �����Ϫ��X�B�����C (�аѦҹ�~\ref{fig:yzu_logo_6} �P \ref{fig:yzu_logo_7})

{\centering\begin{lstlisting}[caption={mpfigt �ϥνd�ҽX},
label=lst:mpfigt,
numbers=left,
firstnumber=1,
frame=ltrb, % single lines for left, top, right, bottom; LTRB for double lines 
%escapeinside={$$}, %�p�n�b�C������ܯS���r��/�ƪ��ĪG�A�n��Ӥ�r��� $$ �]���� (�A�X C �{���X)(��w�]�� <>)
]
\begin{figure}[tbph]
%
\mpfigt{0.25} %width in terms of \columnwidth
{figs/example_fig} %path/filename (no space)
{<�ϥ� mpfigt �i�ܤ������� 3>} %caption
{fig:yzu_logo_6} %label
{<�^���ۤ����j�Ǻ���>} %additional text beneath the caption
%
\hfill
%
\mpfigt{0.25} %width in terms of \columnwidth
{figs/example_fig} %path/filename (no space)
{<�ϥ� mpfigt �i�ܤ������� 4>} %caption
{fig:yzu_logo_7} %label
{<�^���ۤ����j�Ǻ���>} %additional text beneath the caption
%
\end{figure}
\end{lstlisting}\par}

%%%%%%%%%%
% \begin{figurehere} % for non-float
%
\begin{figure}[tbph]
%
\mpfigt{0.25} %width in terms of \columnwidth
{figs/example_fig} %path/filename (no space)
{�ϥ� mpfigt �i�ܤ������� 3} %caption
{fig:yzu_logo_6} %label
{�^���ۤ����j�Ǻ���} %additional text beneath the caption
%
\hfill
%
\mpfigt{0.25} %width in terms of \columnwidth
{figs/example_fig} %path/filename (no space)
{�ϥ� mpfigt �i�ܤ������� 4} %caption
{fig:yzu_logo_7} %label
{�^���ۤ����j�Ǻ���} %additional text beneath the caption
%
\end{figure}
% \end{figurehere} % for non-float
%%%%%%%%%%%%

%%%%%%%%%%%%%%%%%%%%%%%%
\item \verb+\mpfigts+ �P�e�z�� \verb+\mpfigt+ �ۦP�A���O�h�F�վ���D��r�P�Ϫ����Z�վ� (\uline{s}pacing)�C�s���Ѽơu���Z����v�i�H�վ�ϩ��U�P���D������r���������Z�C�q�`�O�]�����Z�Ӥj�ӧƱ��Ԫ�ϻP���D��r�C�Q�Ԫ�Z���A�h�έt�����סA�Ҧp $-1$cm�A$-4$ex (�H�^��r�� x �����׬����)�C�p�G�n�H��l���Z�e�{�A�h�� 0cm, 0ex�C
�аѦҹ�~\ref{fig:yzu_logo_8} �P \ref{fig:yzu_logo_9}�C

{\centering\begin{lstlisting}[caption={mpfigts �ϥνd�ҽX},
label=lst:mpfigts,
numbers=left,
firstnumber=1,
frame=ltrb, % single lines for left, top, right, bottom; LTRB for double lines 
%escapeinside={$$}, %�p�n�b�C������ܯS���r��/�ƪ��ĪG�A�n��Ӥ�r��� $$ �]���� (�A�X C �{���X)(��w�]�� <>)
]
\begin{figure}[tbph]
%
\mpfigts{0.35} %width in terms of \columnwidth
{figs/example_fig} %path/filename (no space)
{<�ϥ� mpfigts �i�ܸ��j���P���D��r���Z>} %caption
{fig:yzu_logo_8} %label
{<����^���ۤ����j�Ǻ���>} %additional text beneath the caption
{-1ex} %place negative height, say, -4ex to reduce the gap between fig and caption
%
\hfill
%
\mpfigts{0.35} %width in terms of \columnwidth
{figs/example_fig} %path/filename (no space)
{<�ϥ� mpfigts �i�ܽվ���p���P���D��r���Z>} %caption
{fig:yzu_logo_9} %label
{<����^���ۤ����j�Ǻ���>} %additional text beneath the caption
{-4ex} %place negative height, say, -4ex to reduce the gap between fig and caption
%
\end{figure}
\end{lstlisting}\par}

%%%%%%%%%%
% \begin{figurehere} % for non-float
%
\begin{figure}[tbph]
%
\mpfigts{0.35} %width in terms of \columnwidth
{figs/example_fig} %path/filename (no space)
{�ϥ� mpfigts �i�ܸ��j���P���D��r���Z} %caption
{fig:yzu_logo_8} %label
{����^���ۤ����j�Ǻ���} %additional text beneath the caption
{-1ex} %place negative height, say, -4ex to reduce the gap between fig and caption
%
\hfill
%
\mpfigts{0.35} %width in terms of \columnwidth
{figs/example_fig} %path/filename (no space)
{�ϥ� mpfigts �i�ܽվ���p���P���D��r���Z} %caption
{fig:yzu_logo_9} %label
{����^���ۤ����j�Ǻ���} %additional text beneath the caption
{-4ex} %place negative height, say, -4ex to reduce the gap between fig and caption
%
\end{figure}
% \end{figurehere} % for non-float
%%%%%%%%%%%%

%%%%%%%%%%%%%%%%%%%%%%%%
\item \verb+\mpfigtsr+ �P�e�z�� \verb+\mpfigts+ �P�A���W�[�F�Ϫ��|�P�ťեi�H���� (t\uline{r}im) ���\��C�۹������Ѽƪ��Ʀr�N�q�p�U�G���B�U�B�k�B�W�A�n���������סA�H bp �����A1~bp $= \frac{1}{72}$~inch�C
�аѦҹ�~\ref{fig:yzu_logo_10} �P \ref{fig:yzu_logo_11}�C

{\centering\begin{lstlisting}[caption={mpfigtsr �ϥνd�ҽX},
label=lst:mpfigtsr,
numbers=left,
firstnumber=1,
frame=ltrb, % single lines for left, top, right, bottom; LTRB for double lines 
%escapeinside={$$}, %�p�n�b�C������ܯS���r��/�ƪ��ĪG�A�n��Ӥ�r��� $$ �]���� (�A�X C �{���X)(��w�]�� <>)
]
\begin{figure}[tbph]
%
\mpfigtsr{0.25} %width in terms of \columnwidth
{figs/example_fig} %path/filename (no space)
{<�ϥ� mpfigtsr �i�ܵ����֪�����>} %caption
{fig:yzu_logo_10} %label
{<����^���ۤ����j�Ǻ���>} %additional text beneath the caption
{0ex} % additional spacing between fig and caption
{5 5 5 5} % left, top, right, bottom trimming, in unit of bp (=1/72 in)
%
\hfill
%
\mpfigtsr{0.25} %width in terms of \columnwidth
{figs/example_fig} %path/filename (no space)
{<�ϥ� mpfigtsr �i�ܵ����h������>} %caption
{fig:yzu_logo_11} %label
{<����^���ۤ����j�Ǻ���>} %additional text beneath the caption
{0ex} % additional spacing between fig and caption
{20 20 20 20} % left, top, right, bottom trimming, in unit of bp (=1/72 in)
%
\end{figure}
\end{lstlisting}\par}

%%%%%%%%%%
%\begin{figurehere} % for non-float
%
\begin{figure}[tbph]
%
\mpfigtsr{0.25} %width in terms of \columnwidth
{figs/example_fig} %path/filename (no space)
{�ϥ� mpfigtsr �i�ܵ����֪�����} %caption
{fig:yzu_logo_10} %label
{����^���ۤ����j�Ǻ���} %additional text beneath the caption
{0ex} % additional spacing between fig and caption
{5 5 5 5} % left, top, right, bottom trimming, in unit of bp (=1/72 in)
%
\hfill
%
\mpfigtsr{0.25} %width in terms of \columnwidth
{figs/example_fig} %path/filename (no space)
{�ϥ� mpfigtsr �i�ܵ����h������} %caption
{fig:yzu_logo_11} %label
{����^���ۤ����j�Ǻ���} %additional text beneath the caption
{0ex} % additional spacing between fig and caption
{20 20 20 20} % left, top, right, bottom trimming, in unit of bp (=1/72 in)
%
\end{figure}
%\end{figurehere} % for non-float
%%%%%%%%%%%%


%%%%%%%%%%%%%%%%%%%%%%%%
\item \verb+\mpfigtsra+ �o�O�\��̧��㪺�h�Ϩñƫ��O�A�N�򥻪� \verb+\mpfig+ �[�W�u��N�ӷ�����r (\uline{t}ext)�v�A�u�P��r���Z (\uline{s}pacing)�v�A�u���� (t\uline{r}im)�v�A�u��� (\uline{a}lign)�v�C�o�s�W������A�O���F�]���p�G�n�ñƪ��Ϥj�p���P�A�W�z���X�Өñƫ��O���|��Ϫ�����P�r���� (align)�A�y�����G���@����ı�ĪG�C�p�G�A�Q�n���O���� (t) �B�y�u (���u c)�B�Ω��� (b) ����r��A�h�i�H�ϥγo�ӫ��O�C
�һݪ��ѼƤ]�̦h�C���r�ѼƪŵۡB���Z�W�q�]�� 0cm�B�����q 0 0 0 0�A�o���O�]�i�H�����\����֪� \verb+\mpfig+, \verb+\mpfigt+, \verb+\mpfigts+, \verb+\mpfigtsr+ �ӥΡC�аѦҹ�~\ref{fig:yzu_logo_14} �P \ref{fig:yzu_logo_15}

{\centering\begin{lstlisting}[caption={mpfigtsra �ϥνd�ҽX},
label=lst:mpfigtsra,
numbers=left,
firstnumber=1,
frame=ltrb, % single lines for left, top, right, bottom; LTRB for double lines 
%escapeinside={$$}, %�p�n�b�C������ܯS���r��/�ƪ��ĪG�A�n��Ӥ�r��� $$ �]���� (�A�X C �{���X)(��w�]�� <>)
]
\begin{figure}[tbph]
%
\mpfigtsra{0.27} %width in terms of \columnwidth
{figs/example_fig} %path/filename (no space)
{<�ϥ� mpfigtsra �i�ܹ���H�μ��D��r���Z���j>} %caption
{fig:yzu_logo_14} %label
{<����^�ۤ����j�Ǻ���>} %additional text beneath the caption
{0ex} %additional spacing between fig and caption
{0 0 0 0} %left bottom right top; in unit of bp (1/72 in)
{c} %vertical alignment: b, c, t
%
\hfill
%
\mpfigtsra{0.3} %width in terms of \columnwidth
{figs/example_fig} %path/filename (no space)
{<�ϥ� mpfigtsra �i�ܵ����B���j���ϡB����H�μ��D��r���Z���p>} %caption
{fig:yzu_logo_15} %label
{<����^�ۤ����j�Ǻ���>} %additional text beneath the caption
{-4ex} %additional spacing between fig and caption
{10 10 10 10} %left bottom right top; in unit of bp (1/72 in)
{c} %vertical alignment: b, c, t
%
\end{figure}
\end{lstlisting}\par}

%%%%%%%%%%
%\begin{figurehere} % for non-float
%
\begin{figure}[tbph]
%
\mpfigtsra{0.27} %width in terms of \columnwidth
{figs/example_fig} %path/filename (no space)
{�ϥ� mpfigtsra �i�ܹ���H�μ��D��r���Z���j} %caption
{fig:yzu_logo_14} %label
{����^�ۤ����j�Ǻ���} %additional text beneath the caption
{0ex} %additional spacing between fig and caption
{0 0 0 0} %left bottom right top; in unit of bp (1/72 in)
{c} %vertical alignment: b, c, t
%
\hfill
%
\mpfigtsra{0.3} %width in terms of \columnwidth
{figs/example_fig} %path/filename (no space)
{�ϥ� mpfigtsra �i�ܵ����B���j���ϡB����H�μ��D��r���Z���p} %caption
{fig:yzu_logo_15} %label
{����^�ۤ����j�Ǻ���} %additional text beneath the caption
{-4ex} %additional spacing between fig and caption
{10 10 10 10} %left bottom right top; in unit of bp (1/72 in)
{c} %vertical alignment: b, c, t
%
\end{figure}
%\end{figurehere} % for non-float
%%%%%%%%%%%%

%%%%%%%%%%%%%%%%%%%%%%%%
\item \verb+\mpfigabc+ ���P��e���X�Өñƪ����O�A�o�ӫ��O�u���ͤ@�Ӽ��D��r�P�s���C�A�X�n�X�Ӥl�ϨñơC�Ьݹ�~\ref{fig:yzu_logo_16}�C

{\centering\begin{lstlisting}[caption={mpfigabc �ϥνd�ҽX},
label=lst:mpfigabc,
numbers=left,
firstnumber=1,
frame=ltrb, % single lines for left, top, right, bottom; LTRB for double lines 
%escapeinside={$$}, %�p�n�b�C������ܯS���r��/�ƪ��ĪG�A�n��Ӥ�r��� $$ �]���� (�A�X C �{���X)(��w�]�� <>)
]
\begin{figure}[tbph]
%
\mpfigabc{0.15} %width in terms of \columnwidth
{figs/example_fig} %path/filename (no space)
{(a)} %text; (a) (b) (c)
%
\hfill
%
\mpfigabc{0.2} %width in terms of \columnwidth
{figs/example_fig} %path/filename (no space)
{(b)} %text; (a) (b) (c)
%
\hfill
%
\mpfigabc{0.25} %width in terms of \columnwidth
{figs/example_fig} %path/filename (no space)
{(c)} %text; (a) (b) (c)
%
\vspace*{-2ex}
\caption{<�ϥ� mpfigabc �i�ܤ��P�ؤo������>} % common caption
\label{fig:yzu_logo_16}
\end{figure}
\end{lstlisting}\par}

%%%%%%%%%%
% \begin{figurehere} %for non-float
%
\begin{figure}[tbph]
%
\mpfigabc{0.15} %width in terms of \columnwidth
{figs/example_fig} %path/filename (no space)
{(a)} %text; (a) (b) (c)
%
\hfill
%
\mpfigabc{0.2} %width in terms of \columnwidth
{figs/example_fig} %path/filename (no space)
{(b)} %text; (a) (b) (c)
%
\hfill
%
\mpfigabc{0.25} %width in terms of \columnwidth
{figs/example_fig} %path/filename (no space)
{(c)} %text; (a) (b) (c)
%
\vspace*{-2ex}
\caption{�ϥ� mpfigabc �i�ܤ��P�ؤo������} % common caption
\label{fig:yzu_logo_16}
\end{figure}
%  \end{figurehere} %for non-float
%%%%%%%%%%%%


%%%%%%%%%%%%%%%%%%%%%%%%
\item \verb+\mpfigabcs+ �P�e�z���O�ۦP�A�A�X�n�X�Ӥl�ϨñơA���O�i�H������O��r (a), (b), (c) �P�Ϥ��������Z�C�Ьݹ�~\ref{fig:yzu_logo_17}�C

{\centering\begin{lstlisting}[caption={mpfigabcs �ϥνd�ҽX},
label=lst:mpfigabcs,
numbers=left,
firstnumber=1,
frame=ltrb, % single lines for left, top, right, bottom; LTRB for double lines 
%escapeinside={$$}, %�p�n�b�C������ܯS���r��/�ƪ��ĪG�A�n��Ӥ�r��� $$ �]���� (�A�X C �{���X)(��w�]�� <>)
]
\begin{figure}[tbph]
%
\mpfigabcs{0.15} %width in terms of \columnwidth
{figs/example_fig} %path/filename (no space)
{(a)} %text; (a) (b) (c)
{-1ex} %place negative height, say, -4ex to reduce the gap between fig and text
%
\hfill
%
\mpfigabcs{0.25} %width in terms of \columnwidth
{figs/example_fig} %path/filename (no space)
{(b)} %text; (a) (b) (c)
{-1ex} %place negative height, say, -4ex to reduce the gap between fig and text
%
\vspace*{-2ex} % adjust the gap between (a)(b)(c) text and caption
\caption{<�ϥ� mpfigabcs �i�ܤ��P�ؤo�������A�ϩ��U�����O��r���a��>} % common caption
\label{fig:yzu_logo_17}
\end{figure}
\end{lstlisting}\par}

\begin{figure}[tbph]
%
\mpfigabcs{0.15} %width in terms of \columnwidth
{figs/example_fig} %path/filename (no space)
{(a)} %text; (a) (b) (c)
{-2ex} %place negative height, say, -4ex to reduce the gap between fig and text
%
\hfill
%
\mpfigabcs{0.25} %width in terms of \columnwidth
{figs/example_fig} %path/filename (no space)
{(b)} %text; (a) (b) (c)
{-2ex} %place negative height, say, -4ex to reduce the gap between fig and text
%
\vspace*{-2ex} % adjust the gap between (a)(b)(c) text and caption
\caption{�ϥ� mpfigabcs �i�ܤ��P�ؤo�������A�ϩ��U�����O��r���a��} % common caption
\label{fig:yzu_logo_17}
\end{figure}


%%%%%%%%%%%%%%%%%%%%%%%%
\item \verb+\mpfigabcsr+ �P�e�z���O�ۦP�A�A�X�n�X�Ӥl�ϨñơA���O�i�H����U�l�Ϫ��|������C�Ьݹ�~\ref{fig:yzu_logo_18}�C

{\centering\begin{lstlisting}[caption={mpfigabcsr �ϥνd�ҽX},
label=lst:mpfigabcsr,
numbers=left,
firstnumber=1,
frame=ltrb, % single lines for left, top, right, bottom; LTRB for double lines 
%escapeinside={$$}, %�p�n�b�C������ܯS���r��/�ƪ��ĪG�A�n��Ӥ�r��� $$ �]���� (�A�X C �{���X)(��w�]�� <>)
]
\begin{figure}[tbph]
%
\mpfigabcsr{0.15} %width in terms of \columnwidth
{figs/example_fig} %path/filename (no space)
{(a)} %text; (a) (b) (c)
{-2ex} %place negative height, say, -4ex to reduce the gap between fig and text
{0 0 0 0} %left bottom right top; in unit of bp (1/72 in)
%
\hfill
%
\mpfigabcsr{0.25} %width in terms of \columnwidth
{figs/example_fig} %path/filename (no space)
{(b)} %text; (a) (b) (c)
{-2ex} %place negative height, say, -4ex to reduce the gap between fig and text
{10 10 10 10} %left bottom right top; in unit of bp (1/72 in)
%
\vspace*{-2ex} % adjust the gap between (a)(b)(c) text and caption
\caption{�ϥ� mpfigabcsr �i�ܤ��P�ؤo�������A�ϩ��U�����O��r���a��A�B�� (b) �|�䦳����} % common caption
\label{fig:yzu_logo_18}
\end{figure}
\end{lstlisting}\par}

%%%%%%%%%%  for npc_macros_20120912
%%%%%%%%%%
% \begin{figurehere} %for non-float
%
\begin{figure}[tbph]
%
\mpfigabcsr{0.15} %width in terms of \columnwidth
{figs/example_fig} %path/filename (no space)
{(a)} %text; (a) (b) (c)
{-2ex} %place negative height, say, -4ex to reduce the gap between fig and text
{0 0 0 0} %left bottom right top; in unit of bp (1/72 in)
%
\hfill
%
\mpfigabcsr{0.25} %width in terms of \columnwidth
{figs/example_fig} %path/filename (no space)
{(b)} %text; (a) (b) (c)
{-2ex} %place negative height, say, -4ex to reduce the gap between fig and text
{10 10 10 10} %left bottom right top; in unit of bp (1/72 in)
%
\vspace*{-2ex} % adjust the gap between (a)(b)(c) text and caption
\caption{�ϥ� mpfigabcsr �i�ܤ��P�ؤo�������A�ϩ��U�����O��r���a��A�B�� (b) �|�䦳����} % common caption
\label{fig:yzu_logo_18}
\end{figure}
%  \end{figurehere} %for non-float
%%%%%%%%%%%%


\end{enumerate}
\clearpage
%%%%%%%%%%%%%%%%%%%%%%%%%%%%%%%%%%%
\section{�{���X�C��}
\label{sec:listing}
�A���פ��s�]�\�g�F�{���ӸѨM���D�C�p�G�n�b�פ�̸����{���X�A�i�H�� \verb+lstlisting+ ���ҫ��O�Ӫ��{�C

{\centering\begin{lstlisting}[caption={�@�q�i�� if-else �Ϊk�� c �{���X},
label=lst:c-example,
numbers=left,
firstnumber=1,
frame=ltrb, % single lines for left, top, right, bottom; LTRB for double lines 
escapeinside={$$}, %�p�n�b�C������ܯS���r��/�ƪ��ĪG�A�n��Ӥ�r��� $$ �]���� (�A�X C �{���X)(��w�]�� <>)
]
#include <stdio.h>	

int main()                            /* Most important part of the program!
*/
{
    int age;                          /* Need a variable... */
  
    printf( "Please enter your age" );  /* Asks for age */
    scanf( "%d", &age );                 /* The input is put in age */
    if ( age < 100 ) {                  /* If the age is less than 100 */
     printf ("You are pretty young!\n" ); /* Just to show you it works... */
  }
  else if ( age == 100 ) {            /* I use else just to show an example */ 
     printf( "You are old\n" );       
  }
  else {
    printf( "You are really old\n" );     /* Executed if no other statement is
    */
  }
  return 0;
}
\end{lstlisting}\par}

�o�̮i�ܪ��{���X�O�q c �y���оǺ����^����\cite{url_c_tutorial}�C�i�H�e�ؽu (���B���B�k�B��)�A���s���P���D���� (�o�̪��s���O�C��~\ref{lst:c-example}, �P�ˤ]�O�� \verb+\ref{}+ ���o��)�A�]���渹�A��K�b�����ɰѦҡC

�q�`�A�P�{���������ܼƦW�١A�����ϥμƾǼҦ��A�ϦӡA�]���`�`�|�]�������u�r���Өϥ� \verb+\verb+ ���O�Ӯi�� (�H���r���r��)�C�o�ӫ��O�A�i�H�Τ@��S���Ψ쪺�r���ӷ��]���r�C�Ҧp�A�b�{���C��~\ref{lst:c-example} �� 8 ��A�ϥΤF�@�Ӥ��ت��禡 \verb|printf| ������ܦr�ꤧ�ΡC 

�p�G�פ�̦��h�ӳo�˪��C���A�h�i�H�b \verb+yzu_frontpages.tex+ �̧�u�{���C���ؿ��v���\��Ѱ��ʦL�C�p�G�n�e�{���O��ӵ{�����[�c�A���O�n�����t��k���S�O���B�A�h�C���i�H��b�����C
%%%%%%%%%%%%%%%%%%%%%%%%%%%%%%%%%%%

\section{���J�ѦҤ��m�s��}
\label{sec:cite}

�{�b�ӥܽd�ѦҤ��m���ޥΡC�����A�N���۳Q�ޥΪ����m��ƾ�z�b Bib\TeX\ ��Ʈw�̡C�C�@���ѦҤ��m�A�i�H�O���Z�峹 (article)�B��Q�|�峹 (inproceeding)�B�� (book)�B���~�פ� (mastersthesis, phdthesis)�B�޳N���i (techreport)�B�M�Q (patent)�B�ާ@��U (manual)�B���� (misc) �����O�C�C�@���A���Ӧ��@�ӿW�S�ߤ@�����Ҧr��A�٬� \texttt{bibkey}�A�u��O�^��r���B�Ʀr�A���i���ťաC�i�H�ۤv�R�W�A�]�i�H�Ѹ�Ʈw�޲z�n��̳W�h�N���R�W�C�ܩ���m��ƥ����A�@�̡B���D���A���M�i�H�Τ���A�u�n�T�w�O�H \texttt{UTF-8} �s�X�Y�i�C

��Ʈw���إߡA�i�H�b���ɦb�����u�W��Ʈw�j�M���m�ɡA���K�U���ӵ����m�� citation ��ơA�פJ�A���޲z�n��A�p JabRef, BibDesk, EndNote ���C�Ϊ̡A�@�Ӧr�@�Ӧr����J��Ʈw�̡C�`�����H�����~�O�G(�o�̶��K�ܽd�L�s�����C)

\begin{itemize}
%
\item Bibkey �]�t�F����

\item �p�G�@�̸s�O�q���m���� PDF �ɰ��B�ƻs�X�Ӫ��A�ܥi��|��аO���P�@���ݤ��P��쪺���O�Ÿ��]�@�_�ƻs�i�ӡC�t�~�A�|�ް_ Bib\TeX\ �~�|���r���]��۶i�ӡF�ЬݤU�@���C

\item �@�̸s�C�ӦW�r�����n�H�uand�v�j�}�A�Ӥ��O�r���C�r���u��Φb�U�@�̪��m��B�W�����j�A�Ҧp�A�Z�o�˺��O���o�{�̦W�s Johannes Diderik van der Waals �Ө䤤�m�� van der Waals�C�ҥH�J�����Ʃm�����S�ABib\TeX\ ��Ʈw�n�H�u\uline{van der Waals, Johannes Diderik} \textbf{and} \uline{author 2} \textbf{and} \ldots�v����J�@�̸s���m�W�C

\item ���m�����D (title) �̦r�����j�p�g�|����ҿ�Ϊ����m�榡 (bst ��) ���v�T�ӧ��ܡC�Ҧp�A���פ�d����Ϊ��O \texttt{IEEEtran.bst}�A���u�O�d���D�Ĥ@�Ӧr���Ĥ@�Ӧr�����j�g�A��L�����|�ܤp�g�C�p�G���m���D�t���@�w�n�j�g���M���W���A�Ҧp OLED, GaN ���A�Х� \verb+{}+ ��M���W���]���A�����O�@�ʡC�t�~�A�p�G���D�̧t���W�B�U�СA�Ҧp�ASiO$_2$�A�ХμƾǼҦ��A�[�O�@�� \verb+{SiO$_2$}+ �ӳB�z�C

\item ���Z�W�٤��ݭn���W�Z�n�A�ӥB�Y�g�n�H�D�Ϊ��覡�Τ@�C�Ҧp�A\emph{Applied Physics Letters} �o���Z�A�N�n�H \emph{Appl.\ Phys.\ Lett.} �o���Y�g��b��Ʈw�̡C�i�H�� Web of Science ��Ʈw�����d�ߦU�j���Z�� ISO abbreviation �Y�g�覡\cite{url_journal_names}�C

\item �n�ޥκ����A���}��J��Ʈw�ɡA�n�H \verb+\url{}+ ���覡�e�{�C���m�����ХΡu���� (misc)�v�C

\item ���Z���m���������u���� (No.)�v�C���Z�b�C�@�� (volume) �̷|�X���n�X���A�ӭ��X���|�k�s�C�]�N�O���A�ĤG�������X�A�O�q�Ĥ@�����������X�~��s�U�h���C�ҥH�A�u�n�Щ��ĴX���A�ĴX���A�N�i�H�F�C
\end{itemize}


�b��Z�̭n���J�ѦҤ��m�s���A�Цb�@�y�ܵ����ɡA���I�Ÿ����e�[�J \verb+\cite{bibkey}+ �Y�i�C�p�G�P�ɭn�ޥΦh�g�A�i�H�h�� \texttt{bibkey} �@�Τ@�ӫ��O�A�p \verb+\cite{key1, key2, ...}+�A�t�Φ۰ʷ|���d���Y�g���B�z�A�p�U�@�q���ĤG�y�C�p�G��y�n�H�ӽg���m�@���D�B�����A�h�H�u���m�v�@���f�t�����ͤ�A���� \verb+\citen{bibkey}+ �ӳB�z (�o�إΪk���������A�����o�@�w�n�p��)�C�p�U�@�q�Ĥ@�y�C

�ھڤ��m~\citen{ieee_dmr_2_50_2002_chou} �����k�A�����祲��������j����\cite{jap_093_1108_2003_kondakov, ieee_ed_50_1830_2003_oriols, Chem.Mater._8_1365_1996_Papadimitrakopoulos, jap_079_2745_1996_scott, jap_087_8049_2000_adachi, jap_089_1704_2001_brutting, jap_089_4673_2001_popovic, synth.met._132_9_2002_nomura, cjk_book, thinfilm_macleod_2001,url_wiki_cv}�A�åΤѤ��N���l�ದ��\�C



%%%%%%%%%%%%%%%%%%%%%%%%%%%%%%%%%%%

\chapter{�Ÿ��P��{��}

�ڭ̦b��~\ref{sec:fig} �P \ref{sec:table} �`�ܽd�F�p��B�z���ϥH�Ϊ���C�b�o�@���A�ڭ̨ӬݲŸ��P��{�����B�z�C

\section{�ƾDzŸ��P��{��}
�b����椺�n�Ѽg�ƾDzŸ��B���l�A�i�H�ο��Ÿ��e��]�������ܦ��C�p \$ax+b=0\$ �|��� $ax+b=0$�F�p�G�n��W�i�ܼƾǤ����A�h�� \verb+\[+ �H�� \verb+\]+ �]�������ܦ��C�p�U
\[
\sum_{k=1}^{\infty}\frac{1}{n^2}
\]

�p�G�����n�s���X�A�h�n��\verb+\begin{equation}+ �H�� \verb+\end{equation}+�A�æb���l���[�W \verb+\label{}+ �����ҡC�p
\begin{equation}
  \frac{df}{dx} \equiv \lim_{\Delta x \rightarrow 0} \frac{f(x+\Delta x) - f(x)}{\Delta x}   \label{eq:diff_def}
\end{equation}
�ڭ̥i�H�b���� (\ref{eq:diff_def}) �ݨ�L�����w�q (�ϥ� \verb+\ref{}+ ���o�����s���A�åB�@�̦ۤv�n�[�A���F�Ϊ̡A�ϥ� \verb+\eqref{}+ ���o�a���A���������s��)�C

�i�H�Ѧҧ��G�����ͼg���m�j�a�Ӿ� {\LaTeX}�n�A�Dzߧ�h���ޥ�\cite{lee_latex123}�A�H�θԺɪ� WikiBook ����\cite{url_latex_wikibook_math}�C�аO���A�b \verb+\begin{equation}+ �H�� \verb+\end{equation}+ ��{�����O���Ҹ̡A���i���Ŧ�C�p�G�n�b�s�边�̤��j��{�����P���������O�A�i�H�ε��ѲŸ��_�Y���Ŧ�A�b��ı�W���P�˪��ĪG�C

�H�U�N�C�|�g�פ�ɱ`�������{���B�Ÿ������D�C

%%%%%%%%%%%%%%
%%%%%%%%%%%%%% table name % 
\begin{table}[tbhp]
\caption{\label{tb:math_faq}% 
�`������{���B�Ÿ������D}% 
\vspace*{-4ex}
\begin{center} 
%\resizebox{0.8\columnwidth}{!} {
\begin{tabular}{lllll}% left-aligned, no vertical boarders
\hline 
�y�z&
���~�����O&
���~���ĪG&
���檺�ĪG&
���T�Ϊk\tabularnewline
\hline
{\footnotesize ��즳�W��}&
\verb+cm^3+&
Error!&
cm$^3$&
\verb+cm$^3$+\tabularnewline
%
{\footnotesize ��즳�W�� (2)}&
\verb+$W/m^2$+&
$W/m^2$&
W/m$^2$&
\verb+W/m$^2$+\tabularnewline
%
{\footnotesize �ۭ������}&
\verb+N.m+&
N.m&
N$\cdot$m&
\verb+N$\cdot$m+\tabularnewline
%
{\footnotesize �W�U�Ф���@�Ӧr}&
\verb+$10^-5$+&
$10^-5$&
$10^{-5}$&
\verb+$10^{-5}$+\tabularnewline
%
{\footnotesize ��ǭp�ƪk}&
\verb+6x10$^{23}$+&
6x10$^{23}$&
$6 \times 10^{23}$&
\verb+$6 \times 10^{23}$+\tabularnewline
%
{\footnotesize ��{���̤�r����}&
\verb+T_{hot}=+&
$T_{hot} = \dots$&
$T_\text{hot} = \dots$&
\verb+T_\text{hot}=+\tabularnewline
%
{\footnotesize �h�Ӧr�����Ÿ�}&
\verb+$sin(x)$+&
$sin(x)$&
$\sin(x)$&
\verb+$\sin(x)$+\tabularnewline
%
{\footnotesize �h�Ӧr�����Ÿ�}&
�ƭȤծ| \verb+$NA=+\ldots\verb+$+&
$NA = \dots$&
$\mathrm{NA} = \dots$&
\verb+$\mathrm{NA}=+\ldots\verb+$+\tabularnewline
%
{\footnotesize ���סB�ū�}&
\verb+30$^o$+&
30$^o$&
30$^\circ$&
\verb+30$^\circ$+\tabularnewline
%
{\footnotesize �ū�}&
\verb+$27^{\circ}C$+&
$27^{\circ}C$&
$27^{\circ}$C&
\verb+$27^{\circ}$C+\tabularnewline
%
{\footnotesize �����������q}&
\verb+$V^'=$+&
Error!&
$V'=\ldots$&
\verb+$V'=$+\tabularnewline
%
{\footnotesize ������}&
\verb+5Hz+&
5Hz&
5~Hz&
{\footnotesize\verb+5~Hz+ �� \verb+\unit[5]{Hz}+}\tabularnewline
%
{\footnotesize ������ (�J�_��)}&
\verb+1.0 eV+&
\parbox{1cm}{�@1.0\\[-2ex] eV}&
1.0~eV&
{\footnotesize\verb+1.0~eV+ �� \verb+\unit[1.0]{eV}+}\tabularnewline
%
{\footnotesize ���u�J�v}&
\verb+5.4~A$^\circ$+&
5.4~A$^\circ$&
5.4~\AA&
\verb+5.4~\AA+\tabularnewline
%
{\footnotesize �L��}&
\verb+1~\mu m+&
Error!&
1~$\mu$m&
\verb+1~$\mu$m+\tabularnewline
%
{\footnotesize �t��}&
\verb+-2.1+&
-2.1&
$-2.1$&
\verb+$-2.1$+\tabularnewline
%
{\footnotesize �d��}&
\verb+10\~{}20 m+&
10\~{}20 m&
10--20~m&
\verb+10--20~m+\tabularnewline
%
{\footnotesize �������μƾǼҦ�}&
\verb|ax+b=c|&
ax+b=c&
$ax+b=c$&
\verb|$ax+b=c$|\tabularnewline
%
{\footnotesize ���Υ��έ^�ƬA��}&
&
\makebox[0cm][l]{����������]���A���������^��}&
&
\tabularnewline
%
{\footnotesize �b�����L�ť� (�~)}&
&
\makebox[0cm][l]{����3km(3,000m)��}&
&
\tabularnewline
%
{\footnotesize �b���B���ť� (��)}&
&
&
\makebox[0cm][l]{���� 3~km (3,000~m) ��}
\tabularnewline
%
\hline
\end{tabular}
%}
% end of resizebox
\end{center}
\end{table}
%%%%%%%%%%%%%%
\clearpage
\section{�ƾǦ�}
�ƾǦ��������Ÿ������O����A�Ӥ��O���@��ƾDzŸ����˥α���r���C�Ҧp�A$H_2O$ �O�����T���A���ӬO $\mathrm{H_2O}$�C�ڭ̥i�H�ϥ� \verb+mhchem+ �M�� (�w�]�O���[�J) �̪� \verb+\ce{}+ ���O�ӧ����G\verb+\ce{H2O}+ �i�H�o�� \ce{H2O}, \verb+\ce{CrO4^2-}+ �i�H�o�� \ce{CrO4^2-}�C�ƾǤ������]�ܮe���C�Ҧp�A�b��W��ܤ�{�������� \verb+\[+ \dots \verb+\]+ �Ϊ� \verb+\begin{equation}+ \dots \verb+\end{equation}+ �ϥ� \\
 \verb|\ce{Cu2O + 2e- + H2O <=> 2Cu + 2 OH-}|
 �i�o
\[
\ce{Cu2O + 2e- + H2O <=> 2Cu + 2 OH-}
\]

�p�G�ݭn�A�ѧ�h���ƾǦ��ƪ����O�A�i�H�Ѧ� \verb+mhchem+ �M�󪺻����ѡA�b \url{ftp://www.ctan.org/tex-archive/macros/latex/contrib/mhchem/mhchem.pdf}
%%%%%%%%%%%%%%%%%%%%%%%%%%%%%%%%%%%

\chapter{�a�ۤv�i�����n�ߺD}
\label{sec:habbit}

���DZƪ����Ӹ`�O�n�a�ۤv���A\LaTeX\ �S��k�������C
	\begin{itemize}
	\item �b����y�l�̥X�{���^�B�Ʀr�B�Ÿ��A��e��P����۱����a��Х[�ťաA���O�J�줤����I�Ÿ��h�O�ҥ~�C�ϥ� \verb+\ref{}+ �N�����s���Ʀr�A��e��]�O�@�˳B�z�C�Ҧp�G
		\begin{enumerate}
		\item �i�H����������q���n��OLabView�C(�~)\quad$\leftarrow$�^��r�e��S���d�ťաC
		\item �i�H����������q���n��O LabView �C(�~)\quad$\leftarrow$�^��r���O����y���A�����d�ťաC
		\item �i�H����������q���n��O LabView�C (���T)
		\end{enumerate}

	\item �b�^�B�Ʀr�B�Ÿ��̪����I�Ÿ��Υb���A���n�Τ�������C�Ҥl�Ьݪ�~\ref{tb:math_faq} ���̫�X�h�C(�o�̪����s���e�ᦳ�d�ťաA�o�����]�O�@�ӨҤl)
	
	\item ���ެO����y�l�٬O�^�B�Ʀr�B�Ÿ��A�@�ߥΥb���A���C���A�����e�A�k�A������A�n���ťաA���D�J����I�Ÿ��C
		\begin{enumerate}
		\item �����q����G����ܾ��]TFT-LCD�^�w�g�O�~�ɪ��D�y�C(�~)\quad$\leftarrow$�ϥΤF�����A���C
		
		\item �����q����G����ܾ�(TFT-LCD)�w�g�O�~�ɪ��D�y�C(�~)\quad$\leftarrow$���A�����e�B�k�A������S���d�ťաC
		
		\item �����q����G����ܾ� (TFT-LCD) �w�g�O�~�ɪ��D�y�C(���T)
		
		\item �o�[�c�O�����q����G����ܾ� (TFT-LCD) �C\quad$\leftarrow$�k�A������J����I�Ÿ��A�����d�ťաC
		
		\item �o�[�c�O�����q����G����ܾ� (TFT-LCD)�C(���T)
		\end{enumerate}

	\item �ƾǼҦ��̪��ťշ|�Q�����A�ҥH����y�l�̴��J�Ÿ��ɡA�ťխn�[�b�ƾǼҦ����~�C�Ҧp�A�G�������Y��$ a_2 $�P�`�ƶ� $a_0$ ������A��̪��ťդ~�O���T���A���M�e�̦b��Z�̬ݰ_�Ӥ]���ťաA���O \verb+$ a_2 $+ �ťզb�ƾǼҦ��̡A���G�O�L�Ī��C

	\item �e�{��{���ɡA�����{�������q�����@�����A�ҥH�A\verb+\begin{equation}+ �P�e������r�϶��������n���Ŧ�C��{������A�p�G�٦���������r�A�]�������P�@�q���A�ҥH�A\verb+\end{equation}+ �P���᪺��r�϶��]���n���Ŧ�C�p�G�ݭn�b�s�边�̯����Ϥ��o�̬O��{���A�i�H�γ�@�Ӧʤ������ѡA�N���Ŧ�C
	
		\begin{equation}
		E = ma^2
		\label{eq:ma2}
		\end{equation}
		
	�o�Ӥ�{���b��Z�̫e�᳣���Ŧ�A�y���T�Ӭq���A�P�峹�޿�p�G���ūh�n�ק�CŪ�̤]�i�H�ݨ�A��{�����W�B���U���d�զ]���ܤj�A�y����ı�����j�C��ڤW�A�쥻�n���F���O
%
		\begin{equation}
		E = mc^2
		\label{eq:mc2}
		\end{equation}
%
�b��Z�̨ϥΦʤ������Ѧb��ı�W�F��P�Ŧ�@�˪��j���ĪG�A���O�o�O�d�F�P�e���P�@�q�����޿�@�e�ʡC

	\item �b�峹�̴��J�ϮɡA�h�@�߻P�e�����q�A�@�w�n���Ŧ�b���ϫ��O���e�P��C
	\end{itemize}

%%%%%%%%%%%%%%%%%%%%%%%%%%%%%%%%%%%

\chapter{�����j�ǽפ�榡�W�d}
����`�ۡm�����j�Ǭ�s�ҾǦ�פ�榡�W�d���ҡn�A�夤�Ҵ�������A\\
�Ԩ��u���� $\rightarrow$ �аȳB $\rightarrow$ �ǥͱM�� $\rightarrow$ ���~/���� $\rightarrow$ �Ǧ�פ�榡�W�d�v\\
\textless\url{http://www.yzu.edu.tw/admin/aa/index.php/content/view/140/241/lang,tw/}\textgreater\\
���K�ܽd�W���������C \verb+\description+ ���ҫ��O�C

\section{�W�满��}
\begin{description}
\item[�ʭ�]  �аȳB�Τ@�榡�˥��A�p����@�A�g�Ш|���֥i���ժ̬O�_�[���էO�Ϋʭ��ȥѦU�Ҧ۩w�C
\item[�ѦW��] �]�A�פ夤�^��W�١A�۪̤Ϋ��ɱб¤��^��m�W�B�զW�B�ҦW�B�Ǧ�פ�O�B���e�פ�^�廡���Φa�W�A���e�~�뵥�A�p����G�C
\item[�f�թe���|�f�w��] �аȳB�Τ@�榡�˥��A�p����|�A�����ѦU�ҷJ��e��аȳB�A�v���˭q��פ夺�C
\item[���v��] �L�׬O�_�P�N�}��dzN�Q�ΡA���������˭q�A�p����T�A�ûP���n�q�l�ɮ׶}��ϥ�����F�@�P�C (�t�����|���v��) 
\item[���^��K�n] ���e��������s�ت��A��ƨӷ��A��s��k�ε��G���A�� 500--1000 �r�A�åH�@�������C��������K�n�έ^��K�n�A�����Ѽg�A�榡�p���󤭡B���C
\item[�פ�ؤo�ίȱi] �H $210\,\mathrm{mm} \times 297\,\mathrm{mm}$ �W�� A4 �ȱiµ�s�C�ʭ��ʩ��ĥ� 150 �S�H�W�����ȩΥd�ȡA�C�⧡�ѦU�ҫ��w�C
\item[�����W��] �ȱi���ݯd�� 3.5~cm�A�����d�� 4~cm�A�k���d�� 2~cm�A���ݯd�� 2~cm�A�������� 1~cm �B��r�������߽u�Bµ�������C
\item[��r�W��] �峹�D��H���嬰��h�A�ѥ��ܥk�A����rµ�ơA��y���ޥΤ��~�y���H~(~)~�������C
\item[����] (1) ����K�n�ܹϪ��ؿ����A�H i, ii, iii, \ldots\ ���p�gù���Ʀr�s��s���C�ѦW���B�f�w�����L���L�X���X�A�������s�J�P�������X�C\\
(2) �פ夤�Ĥ@���H�ܪ����A���H 1, 2, 3, \ldots\ �����ԧB�Ʀr�s��s���C
\item[�˭q] �۽פ奻���ݸ˰v�A�ѭI�~�K����A���L���~�ŧO�B�Ǧ�פ�O�B�פ�W�١B�աB�|�B�ҦW�B�۪̩m�W�C (������Q�K)�C
\end{description}

\section{�ާ@�ӫh}

\begin{description}
\item[�ؿ�] �����W�d�ҭq�u�פ�s�L���ئ��ǡv�U�����ǡA�̦��s�ƽפ夺�U���ئW�١B���B�`�s���B������ (������E�B�E��1�A�лP���ɱб°Q�׾ܤ@�ĥ�)�C
\item[�Ϫ��ؿ�] �夺���ϡA�U�����ζ��ǡA�������`�s��s���A�ê��C�@���ئ� (������Q�B�Q�@)�C
\item[�Ÿ�����] �U���`���ҨϥΤ��ƾǤίS���Ÿ��A���������C�@�������A�H�K�Ѿ\�A�����U�Ÿ������s�� (������Q�G)�C
\item[�פ奻�峹�`�s��] �����ϥΤ@�B�G�B�K�K (�βĤ@���B�ĤG���K�K) ������Ʀr�s���A�`�q�s���h�t�X�ϥΤ@�B���Т��B���Т��Т��B1.�B (1) �B (�βĤ@�`�B�ĤG�`�B�ĤT�`�B���B�@�B1�B(1)) ���h�����Ǥ����ԧB�Ʀr�C
\item[�פ奻�峹�`�W�٤άq���h��] (������Q�T�B�Q�T��1�A�лP���ɱб°Q�׾ܤ@�ĥ�)\\
(1) �����B���W�٦�󥴦r�������ݤ����B�C\\
(2) �`���B�q�����۪������ݱư_�A�U�Ť@�B�G��Aµ�ƦW�١C\\
(3) �p�q�H�U�������ΦW�١A���H�歺�żƮ涡�Z�����h���C
\item[�פ奻���Z] ���嶡�j�@��A�C���̤� 32 ��A�^�嶡�j 1.5 �� 2 �� (1.5 space or double space)�A�C���̤� 28 ��A���W�U�d������Z�C
\item[�פ奻��r�Z] ���嬰�K���r�Z�A�p���W�d�ϥΦr�Z�A�C��̤� 28 �r�A�^�夣��C
\item[�פ奻���y���Ʀr�B��] ���U�C�`�N�ƶ�\\
(1) �y�z�ʡB�D�B�⤧²��Ʀr�Τ��ƼƦr�A�H����Ʀr���ܡC�ҡG�@�ʤ��Q�H�A�T�U�G�a���A���Q�����Q�C���C\\
(2) �c���̵����p�ϥΤ���Ϊ��ԧB�Ʀr�A�H²�����y�C�ҡG�����T�Q�T���� (���� 3,300,000,000 ��)�C\$15,349 (���Τ@�U���d�T�ʥ|�Q�E����)�C
\item[�פ奻���{���Τ���] ���U�C�`�N�ƶ�\\
(1) ��{���Τ������Y��ƦC�A�ûP���徨�Ϲj���}�C�Ҧp�G
\[ x = \frac{-b \pm \sqrt{b^2 -4ac}}{2a}. \]\\
(2) ��{���Τ������u�@�Ӯ����s�ƧǸ��A�ӧǸ����H��A���е���̫�@�檺�̥k��C�Ҧp�G
\begin{equation}
E = mc^2.
\end{equation}\\
(3) �b���崣���������{���Τ����ɡA�������Ӥ�{�����W�٤ΧǸ��C\\
(4) �p�G��{�����������ɡA�u��b�[�B��B���B���B�j��B�p�󵥹B��l�Ÿ��B���C�W�U�����i��b�����u=�v�B����C�Ҧp�G
\begin{equation}
\begin{split}
f(x) &= f(0) + f'(0)x + \frac{f''(0)}{2!}x^2 + \frac{f'''(0)}{3!}x^3 +\cdots \\
    &=1+\frac{1}{2}x+\left(-\frac{1}{8}\right)x^{2}+\left(\frac{1}{16}\right)x^{3}+\cdots.
\end{split}
\end{equation}\\
(5) ���夤���������ϥ� ``/'' �H�Ϥ����l�P�����C�Ҧp�G1/2�C
\item[�פ奻����}] ���U�C�`�N�ƶ�\\
(1) �S���ƶ����I���A�i�ϥε��} (Footnote) �����C\\
(2) ���}���̶��ǽs���A�s���Щ������k�W��%
\footnote{�o�O footnote ���Ҥl�C}%
�H�ưѾ\�C�U�����s���s��A�U���������۱���C\\
(3) ���}���X�Τ��eµ��P�����ݪ������A�P���大���[����u�Ϲj�A���������i���Φ������ݪ����C
\item[�פ奻����m�Ѿ\] �夤�Ҧ��ѦҤ����m�A�������^��γ��`�A���̰Ѿ\���dzs��s���A�ñN�Ѿ\�s���A�[���A��~[~]~���Щ���Ѿ\�B�C���m��ƥt�s����פ奻�大��C
\item[�פ奻��Ϫ��s��] ���U�C�`�N�ƶ�\\
(1) �����Ϊ��W�C����W��A�ϸ��ιϦW�m��ϤU��C��ƨӷ��λ����A�@�߸m����ϤU��C\\
(2) �Ϫ�����Ʀr�������r�ΥH�u�{�r�Ѽg�C
\item[�ѦҤ��m��ƽs��] �Ҧ��ѦҤ��m��ơA���m��פ奻�大��A�W�ߥt�_�@���A���Ѿ\�s���̦��s���A�������H���屵��C
\item[����] �Z�ݤj�q�ƾڡB���ɡB���������Ψ�L�����Q�Ѥ���ơB�Ϫ��A���i���O�t�_�@���A�s���U�����C���夤���ޥΤ��ѦҤ��m���C�X�ѥظm����������C
\end{description}

\chapter{��g�u��}
�פ�榡�W�d���ǭn�b�@�����ĤG������A���ǭn�b�㭶��r�����p�U�A���ǭn�b�h�h�����`�����p�U�~�ݱo�X�ӡC�ҥH�A���W��g���P���u��A�Ӯi�ܫe���U���`�٨S���J�쪺�פ�W�d�C

\section{�n��W���½þ�ĵ�X�F�~}
\label{speed}
(���@�̬��F�~�A���� \textless{}\url{http://www.1001yeah.com.tw}\textgreater{})

���~�¾�s�~�e�A�ڶi��F��~�Ӫ��Ĥ������q�Ȧ�C���L�o���J�W�F�j�·СI�]���T���b�n�������̰��I���a�����C�o������ƥ����ګD�`�����I�@�B�ک~�M�b�@�ӱ���T�Q�ת��I�ɩY�W����I�G�B�T�����㤧�B�A����~�M�L�k���q�I�T�B�N�����i�H���q�A���]�Sԣ���ΡI�]���n��ثe�O�T��j�����i�J�A�ҥH��Q���ڥ��L�k�}�i�ӱϧڡC���ɧڪ�ı�a�{�����ӬO�T���S�q�B�~�|����I�ҥH�ڴN�}�l�˥i�R�B���b�����d���B���ڱ��q�C���O�����D���d���I�b�T�����㤧�᪺�b�p�ɡA�n����G�D���ʳ������I�~�M�S���b�x�T���g�L�H�b�p�ɤ���A���Ĥ@�x�T���M�J��î�ɡA�N���s�ۥX�{�F�Q�X�x�T�����e�ӨӡA�C�@��r�p����ڪ��������O�@�˪��G��C���t���ԤU�������ݬݧڪ����l������T��������ڷn�n�Y���[�t���h�C���@�p�ɤ���A�~�X�{�F�Ĥ@�x�@�N���U�ӡB���ڦ������l�C���D���ߦa�⨮�Y�˹L�ӡB���ڱ��q�C���G��ڪ��³J�T���̵M�L�k�o�ʡI�ڷQ�o�U�l�V�|�F�I�ڥi��n�ť��󨮰k�`�F�C

\subsection{�o�ͤF�����?}
���L�o�ɩ_�ݥX�{�F�I���s�S�ӤF��x�T���B�۰ʦ۵o�a���U�ӡB�[�J�ר�����C���C�������ɶ}�l���x�F�_�ӡA�U�p���n����W���F�|�x���B�Q�ӤH�@�_�C�L�K�ަa�Q�פF�ר���סC

\subsection{�ר���k}
���n�o�Ǧn�ߤH���Y�A������q�x���h��ר��L�C�ھڥL�̱M�~�������ӬݡA�L�̤@�P�{�����P������ż���F�I��O�j�a�Τ@�ǫܡu���\���v���J���覡�A�����S���a��ڪ�������}�A�νå͸_�l���۽åͯȡA�����䤤���@�Ӥ��P��%
\footnote{����|��ܨ��@�Ӥ��P��ӭײz�O�H��ӧڤ~���D�L�̬O�Mé���I�]���x�n���ר��t����i�D�ڡA���@�Ӥ��P��O�ڨT���|�Ӥ��P�뤧���A
�ߤ@�S���G�٪����@�ӡI}%
�A�M��ڪ��T���N�o�򯫩_�a�o�ʦ��\�F�I

 ���ڪ��T�����\�o�ʤ���A�o�����ڭר����n�ߤH�̬�M�ݤF�ڦѱC�@�Ӱ��D�G�u�аݧA���k�ͶܡH�v�C�ڨ��۹ꪺ�ѱC���ۧڡB�^���D�G�u�ڨS���k�͡A���O���D�O�ڦѤ��I�v�C��O�o�s�n�ߪ��j�k�ʹN���Ĥ@�n�A�T�x�n�ߨ��B�K��n�ߤH�N�o��@���������L�ܡC
 
\subsection{�t�@������}
���L�ڪ��t�@�Ӧ���b�T���o�ʦ��\����~�n���C�o�Ӧ���N�O�G�ڪ��ɳt����C��T�Q�����B���M�N�|���ڪ������A�׺����C�p�B�ک��㪺���q�Z���x�n�٦���ʤ����C�ӥB���U�Ӫ����q�i�O�p�j����B�����U���I�ӥB��V�q�������q�]�O���֡I�o�ئ���X�G�N����O��V�����ҹ��t���u�½þ�ĵ�v�@�ˡG�N�O���t�@�C�A���u�ߧY���z�C
 
�ҥH���U�Ӫ��s���A�ڳ��O��������a�V�e�g�b�C�J��e�観���A�X�G���O���y�]�@�˦a�r����z�B�����W���C�p�G�ݨ�e��O��V�q�������q�A�ڳ��O�w��۪o���B�ζW�L�@�ʤ������t�פ��t�q�L�C

��Ӧh�p�ɤ���A�ڴN���\�a�t��x�n�Y�ר��t�B�`�p�}�F��ʤ����B�U���F��d�E�ʤ��ءB�����U�s�ɳt���E�Q�����C�ڪ��ս��P��Ŧ���D�`�a�k�h�I (�]�����t�u�n�C�󤻤Q�����A�ڪ��³J�T���N�|�o�X�@�ثD�` Weak ���n�T�A�����ڡG�u�D�H��D�H�A�ڤS�n�����o�I���ָ��ʰk�ͧa�I�v�A���ڤߤO���I) 

�½þ�ĵ���G�ƴN����o�̡I���L�b�o�ӬG�Ƹ��Y���@�Ӥp�����A���ڷPIJ�}�h�I�ڳo�Ǥ�l�H�ӡA�ȩ]�ڰj�ɸg�`�|�Q�_�C�e�z��r���O���g����A���ڨT�����㤧�᪺�@�Ӥp�ɤ��A�j�����Q�X�x�T���g�L�I���O�L�̹�ڤ��z���B�B�]���@�N��ڦ��X����A���ڥR���F���n�B�j�ĥ@����U�B�H�ߤ��j�C�N�o�Ӱ��D�A�ڷ��ɴN�ݤF���ڭר������X��~���H�G�u�������~���Ǩ��D�p���S���P���ߡA���@�N���U�����کO�H�v�b���`�����p�U�A�o�Ǧ~���H���ӷ|�^���ڡG�u�F�I�L�̤S���{�ѧA�A�L�F��n���U�����A���q�O�H�A���o����ӶܡH�v�C

����o�ݡB����V�ۥL�����ӷ|�^���ڪ����סC�]���ڤ@��ı�o�o�ǵ��ߤH�h�@�N���U�����ڱ��q�A���ӬO�ݨ�F�ڨ����R�S�i�R���ѱC�����i���a���b���Ǵ���a�H���L�o�Ǧ~���H��O���j���z���H�A�L�̩~�M�^���F�@�ӧڷN�Ƥ��~�����סC�~���H�Z���b�}�i�D�ڻ��G�u�A�����D�p�����j�����q�A�p�����F�����W�j���������~�A�ӥB�ۤv���p���i��]�|�]���S�q�O�I�v

�t�~�@�Ӧ~���H���ۻ��G�u�ä��O�O�H�S�ߨx�B���@�N���A���q�I�ӬO�L�̬ݧA���T���o��j�x�B�˹��@����I�`�Ȧۤv���p�����������W�A�����A�ٷ|���ۧA�@�_�b�n����W����O�I�v�ڷQ�F�@�Q�I�~���H���޿����ӬO���T���I�]���e�����Q�X�x��ک��U���T���u�����O�� March�BFestival �������p���A�ҥH�ڶ}�l�S�X�u���u���ܦ��C�I�v�����������C

�ںݸԵ۳o�x���ڥX�ɪ��ѷݨ��A�گu���{���~���H�Ҩ��ݹ�I�]���ڳo�x�ѷݨ��A���M�u�� 2000~cc�A���O�~���D�`���y�u�B�ӥB����_�y�Ρz�L��A���G�O�ھڧڪ��y�L�Ī��y�z�Ҷq���q�@���I�ڷQ�j�������H�˹L�ڪ��Ψ��@���A���ӻ~�H�����O�@�x 3000~cc �H�W���s�~�j���a�H�~���H�ݧڦp�����C�A�N�}�l���y�p�]�F�_�ӡG�u��ڡI�O�H�٥H���A�O�}�u�L�� (�s�~�j�����N���@) �v�O�I�N�Ȧۤv���U�����A���q�A���������F�L�֡B�ٲ_�������u�̤D�} (�x�y�G�y���@���ޡz���ӭ�) �v�C

���ڨ�{�b�A�٬O�L�k�ҹ�u�p�����j�����q�A�p�����������W���A�ӥB�i���ٷ|�`�F�ۤv�I�v�����k�O�_�ݹ�H�n�a�I�ݨ�o�X��n�ߦ~���H�����W�A�ڴN���o�ӻ��k���u�I

�o�ӡu�����F�L�֡B�ϦӦ����̤D�}�v���z�סA���گu�O�PIJ�}�h�I�ר�O�b�ڪ��C�b�\�U�Y�N�������ڡC

�O�o�h�~�E�멳�A���C�b�\�U�o�ͤF�i�Ȫ��u�E�G�|�G��%
\footnote{�ҿת��u�E�G�|�G�סv�A�Ш������I}%
�v����A�ڪ��H�ʹN�H�۳o���b�Hť�D���G�פ@�P Down �J�F�`�W�C

�h�~���Q���A���ڱN�o�a�w�g���}�ť����~�B�w�g�Q�H�t�\�w�e���C�b�\�U���s�}������C���|��F�谩�ʤߡB���ڥûx���Ѫ��W�G�I�]���y���Ӵ_�͡z���C�b�\�U���w�g�Q���H (�ר�OŪ�̪B��) ���B�~�Z�˸ѤF���C�B�K���B�ӥB�@���b�C�ɱr�ޡB�@�L�_�d����H�C�ҥH�b 2004 �~�~�������T�Ӥ�A�ڳ��J���`���~�{���A�ھa�۰����q���ܼ~�{�Ħb�L���C���ɧ��`�O�Q�ۡG�ک����O�y�E�G�|�G�סz���̤j���`�̡A���O������٭n���۳o�ءu�S���۫H�v���o�W�B�W�W�L�k½���O�H�b���q�ɶ��A�ڪ��D��N�P�ڲ������H�D�`�a�h�I�ܦh�C�b���ȱq���A�]���@�N��J�C�b�j���I�ڨ��ӭ覨��Ԫ��Ѥ� Yeah �¡A�@�ǯਥ���D���i�N�]�ɯ�¶�]�B�y���׾ª��j����C�ڬO�@�ӷPı�ܲӷL���H�A�ڥi�H�ܲM���a��ı�ڪ��H��B�ڪ��ζH�B�ڪ��۫H���w�g�H�ۡu�E�G�|�G�סv�˸ѡI�ҥH�h�~ Q4 ���ڤ@�����b�L�����W�P�L�U�����ҤU�B�N���@�ӷĤ����H�b���̯B���B�I�A�止�a���ۤ�D�ϡI�p�G�Χڪ��n��G�ƨӸ����ڷ��ɪ����p�A�ڷQ�ܦh�H���H���ڬO���ݱ��q���s�~�j���B�`�Ȧۤv���p�����������W���A�٭n���ۤ@�_����B�ӱo���v���I�u�����A�A�γ\�`�F�A�I���O���A�A�o�i��`�F�A��ڡI��W�O�H�v�ڷQ�o�O���ɫܦh�H��ڪ���ı�����a�H�zı�o�ڦb�d�Ƿ����ڦӥh���s�j�H�s�ܡH���I�گu�����Q�d�ǥ���H�I�ӥB�ڤ]�S���v�Q�d�ǡI�]���ڭ̴N�N�ߤ�߹��I�p�G�ڬݨ�ڥ��� (�魫���ڨ⭿�� NBA ���P) �Ĥ��A�ڷQ�ڤ]�������X����I�p�G�ڬݨ�@�x���h 600 ����b�n��B���ݧ��������q�C�ڤj���]�|�򻫤h���D�S�X�@�y�L�`�����e�B�߸̷Q�ۡG�u�z�Ѯv�d�n�աI���A���q�A���ڪ��Ψ��@�w�|�ߧY����b���ǡI�@�ѤѫO���A�A��������B�U�ֺ��R�ȡB�����I�v�A�M��[�t�p���{���C���ڭ̤@���ŷR�i�D�O�H�G�u���n�ާO�H���Q�I���A�ۤv�̭��n�I�v�C���I�ڲ{�bı�o�o�y�ܨä���I����z�������b�o�ӥ@�ɤW�A�z�o�i�H���n�b�G�@�ɤW����L�H��A���ݪk�O�H�ѩ�j�a���p�R�P��R�A�γ\�ܦh�H���{�������ӬO�@���p�����@�ˡB�ۥߦ۱j���H�ߡB�e�~�����S���ꪺ�@���H���C�γ\�ڳQ�~�H���O�@���魫�����ʤ��窺�ΨСB�άO�@�x cc �ư��� 4000 ���j���C�ҥH���ڷĤ��B���ڨS�q���ɭԡA�j�a�O���Ӵ����ڡA�`�������W���B�٭n�@�_�����C�{�b���ڤw�g�׹L�F���e�~�{�L�񪺳�ŵ�A���ާڪ��C�b�\�U�~�Z�˦��F��㤭�Ӥ�@�L�ϼu�C���L�O�ڮ��ĵL�񪺦n�����O�G�H�ᤣ�ΦA�W�W���ݺC�b���~�Z�ϼu�F�A�]���C�b���W�N�n�U���F�IYeah�I�ܩ�h�~���߱�ڪ����h�B�͡BŪ�̻P�ȤH�A�ڤ]�b�ǵ۳o�����}�q�l�����F�u�j�j�v���N�C���L�o�Ӥj�j�A�ä��O�ڭn��̩߱�ڪ��H�A�ӬO�Ʊ�߱�ڪ��H�i�H��̧ڡA��̧ڦb�Ĥ��P����ɪ�������A�C�P�¦U���ڷQ���@���H���A�ҥH�~�|�b���ɹ�ڿ�ܡu�S���S��v���B�m�覡�C�ڨä��O�G�N�b���ϸܡA�ӬO�ۤ߸۷N���P�¦U��A�]���z��ڪ��Q���A�O���ڥ��ӵo�i���̤j���R�I�z�ŰڡI



\section{�_�L�y���ѸݺM���סX�F�~}
(�b�� \pageref{speed} ���� \ref{speed} �`�̧ڭ̴���F���@�̻P�X�B�C)



�e�X�ѡA����e�`�ά_�L�y���ͨӥx�X�ݡA�U�j�C�鳣�����a���ɵۡu�P�_�L�y����A�n�I�x���@�U���H�I�v���s�D�C�ӥB�ڭ̦b�q���W�ݨ�F���������W�H�������b�P�_�L�y���⪺��C�����C

�u���Ӥ�n�@�U���H���N���Y�O���O�n�T�U���H���ڥΤO�a��_�L�y�����ѬO���O�n�@�ʸU���H�v���g�ݨ�o�q�s�D���B�ͻP�ȤH���C�L�K�ަa�P�ڲ�ۡC

���M��ƹ�ä��O�p���I�ϥ��x�W���C����w²�Ƽ��D�B�[���]���w����_�����q�A�~�|�X�{�o�ءu���Ӥ�n�@�U���H�I�v�������s�D���D�C

�@�U�����ä��O�I���_�L�y���H�A�ӬO�I��XX�X�����C�]��ú�F�o�@�U������A�z�N�i�H�����ӥX�������L��|���A�@�~�i�H���o���Ȥ@�U���d�����s�ѡC�S�i�H��_�L�y����A�٥i�H�ݨ쿽�����j���k�b�|������C�z��ı�o�o�O�@�ӫD�`�E�⪺�ӷ~����ܡH�p�G�_�L�y�Ӫ����@�ѡA�ڤ��O���n�b���۷h�a�A�����ӷ|�[�J�P�_�L�y�����⪺��C���C

�b�ڪ����̡A�_�L�y���ͬO�@��ȱo�j�a�Dz߻P�Īk�����\�H���A���ޥL�ѥS���g�ǤU��Ѱʦa����z�ƥ�C�����M�O�夣����C

�̪�ګܳ��w��B�ʹ��_�_�L�y���@�ӬG�ơC�o�άG�ƬO�o�ͦb�_�L�y���b�M�D�`�γs����1996�~�A���ɦ��@�ӳy��t�u�|�ܽЬ_�L�y�ӳX�A���K�i�H�ԩԲ��I

�o�a�y��t�A�b���Q�X�~�ӥͷN�@���ܮt�A�Q�n�����G���y��t���o�줣���x�I�J�Y���B�����ܷQ�ӭ������j�N�A���O�s�j���Ҥ軡����]���@�N�I�Ҥ�{���L�̦b�o�ӳy��t�����F�o��[�B���Z�ण���L�̦h�~�Ӫ����W�İ��B�����N���O�H�ҥH�y��t���u�|�K��W�F�M�D�s�����_�L�y�A�Ʊ�_�L�y�i�H���U�L�̪����B�O���y��t�����I

�H���`���F���޿�ӬݡA�_�L�y�j�i�H�D�`�E�ʡB�f�j��a�i�D�u�H�̡G�u�U��m�˰ڡI�p�G�ڳs������A�ڥi�H��U��O�ҡA�_�L�y�M�w���|���y��t�����A���U��m�˥i�H�~�򬰳y��t�����İ��A�A�̻��n���n�ڡI�v�C�M��P�򪺹����H���N�|���K���_���J���j�ۡG�u�_�L�y���[ (�x�y�G���蠟�N) �_�L�y���[�v�C

���L�_�L�y���ɨèS����ܤW�t�o�ؿE���t�X�K�K�C

�_�L�y���ɬO�o�򻡪��G�u���ާڥ��ӷ���P�_�H�ڥ����n�Z�Ӧa�i�D�U��m�ˡF�Q�u�t�O�D�����i�I�]���Q�u�t�ڥ��S���Q���B�S�S���e�~�I�U��m�˫ݦb�o�a�y��t�u���Sԣ�d�Y�I�ګ�ij�U��m�˭̳̦n�O��欰���I���n�A���y��u�H�F�I�v

�_�L�y�����@�X�A�x�U�s�j���Ҥu�B�ͦ۵M�O�N�n�|�_�B�F���n���Ķ��]�I�N���u�Ѥl�ڡ�ť�o�ܤ��n�I�v����ij���J�]�w�g�dzƦn�B�W�իݵo�I���L�_�L�y��w�����a�~�򻡹D�G�u�p�G�ڶ��Q�����`�Τ���A�ګO�ҬF���@�w�|�d�@�a�ɲ߾ǮաA�K�O���U��m�˭̧K�O�Dz߷s�ޯ�C�ӥB�ٷ|�q�ȩ����U�U�컲�ɴN�~�C�A�̻���n���n�I�m�˧r��v

��ӳo�s�m�˭̨쩳�S���벼���_�L�y�H���u���I�ڨä��ӲM���I�ڥu���D���~�����`�Τj��A�_�L�y�O�զp�}�˦a�j�ӡC

�o�a�Q�_�L�y�ݰI���y��t�b�_�L�y�s������N�����j�N�I���L�_�L�y�]�I�{�F�L������m�˪��v��ӿաA�]������F���u���K�O���o�dzy��t�u�H�h�Ƿs�ޯ�B�ӥB�]���ɥL����~���\�C

�ګܪ��D�z�q�o�Ӥp�G�Ƥ���o�F����ҥܡH���L�ګܷQ�i�D�A�A�̪�ڤ��ҥH�R���_�o�ӬG�ơA�ä��O�Q�i�D�j�a�_�L�y�O�Ӹ۹ꪺ�H (�_�L�y�p�G�u���ܸ۹�A���L���ѱC�G�ƩԿ��F��o�����L�O�H)�C�ڧ󤣷Q�Φ��G�ƨӮ��v�ۭ��B�ӷt�إx�W���F�v���ҡC

�ڱq�o�ӬG�Ƹ̡A��o�@�ӫܭ��n���i���C�_�L�y�O�ӵ����o���B�D�`���\���H (�کl��ı�o����s�������`�Ϊ��H�����L�H���B�A�ʤ����ʬO�������\���H�I���F�{�����p���Ƥ��~)�C��ı�o�b�_�L�y���H���޿褤�A�@�w�]�t�F�u�IJv�v�o��r�A�ӥB�IJv�����n�����K�w�b�_�L�y���H�B�ƪ�����e��W�C

�y��t�S���d�Y�B�����L�n���I�Ӥ��Ӧ��H�_�L�y���G�u���I�v

�y��t���v�y�[�B�O�o�ӫ��������n�믫�H�x�A�ӥB�o�a�y��t�H�e���g���ȿ��A���O�{�b�w�g�L���@�d�Y�A�z���Ӥ��Ӧ��H�_�L�y���G�u�����I�٬O�n���I�v

�y��t���M�S�d�Y�I���O�A��򻡦o���X�U�W���u�A�o�X�U�W���u���˪B�n�ֻͤ��]���Q�ӸU�H�A�ӥB�h�b����_�L�y���ݪ����D�ҡC�j�H�ڡI�z���Ӥ��Ӧ��I�_�L�y�K�F�K���Y�B�`�F�`�f���A�٬O�|�j�n�a���G�u�u�n�S���IJv�B�S���d�Y�P�e�~���Ʊ��A���n���I�v

����O�@�ɤW�D�`�Ѧr�����Y���ꥻ�D�q��a�A�|���i�X����޿�ҥH�u�IJv�v���D�B�u�{��v���骺�_�L�y�A�o�ä����H�P��N�~�I�ӥB��ı�o�N��H�v�в����Ӭ� (�L�׬O����٬O�����)�A�IJv�]�O�������n���Ʊ��I

�@�ӨS�d�Y�B�S�IJv�A���ȿ����Ʒ~�A�խY�u���a�v���~��g��U�h�A�u�O�����Y���H (�L�צ���P���u) ���n�Į�B���H�ͷP�쵴��C�ҥH�o�بƷ~�@�w�n���I�K�o�ˮ`��L�d���H�B�L�d�����|�P���|�C

�p�G�_�L�y���ͯu���@�N��ڮi�}��ܡA�ڲq�Q�ڭ̪���ܷ|�O�p���C

�u�_�S�A�аݧڪ��C�b�\�U�Ӥ��Ӧ��H�v�ڰ@�ۦa�ݹD�C

�u�S�d�Y�I�N���I���ȿ��I�]���I�S�e�~�I�٬O���I�A�g��o�a�\�U�ֶּܡH�p�G���ּ֡I���@�w�o���I�����F�A���¡A�ӡ�U�@��I�����p�j�O�ܡH�v�����IJv���_�L�y�p�s�]���@��a�����C

��O�ڤSú�F�@�U���A�S���s�ƤF�@�����A�S��L���F�@����A�S�o�ݤF�@�Ӱ��D�C

�u�аݬ_�j�v�A�аݧA�{�������\�w�q����H�v�ڨ̵M�@�ۦa�ݡA�u�t�S������X�Q�C

�u�ڭӤH�{�����\���w�q�N�O�G�y�b�A���w�B�ժ����ƪ��W�A�D�`���ԧ�J�B�D�`�o�����IJv�a�u�@�ۡI�z�C�����F�A���¡A�U�@��A�J�ا¤p�j�C�v�{�ꪺ�_�L�y�̵M���IJv�a����\��F�C

���U�ӧڴN���@�N�~��o�ݤF�A�]���ھ�ߧڪ����Y�٦��L�Ӭ¡B�����o�P���R�����k�̦b�ƶ����۸�_�L�y����C

�ڱq�_�L�y���j�p�G�Ƥ��A�ڱo�F�D�`�h�öQ���ҵo�C�ڨä��|�@�N�a�{���_�L�y�O���بS�ߨx�B�S��\���F����u�A�̥ͦs�v�����l�C�]����ı�o�a�y�̦n���B��覡�A����O�̦��IJv���B��覡�C�p�G�ڭ̥h���a�@�ӨS���e�~�B�S���d�Y�P�IJv���Ʒ~�b�a�y�W�~��s�b�A�Φ�a�����J�ӻ��A���ڥ��N�O�@�إi�c���y�~�欰�C

�b�����Y�A����Ӧr���o���D�`�ۦ��I�@�ӬO�M�� (�A�o�����utesyu�v�A�N��򤤤媺�N�䧹���@�ˡI�b�����`�ءG�Ʋz�F��x�g�`�|ť��I) �t�~�@�Ӧr�O�Ѹ� (�A�o�����utensyu�v�A��Ѹݦa�����y�Ѹݡz�N��@�Ҥ@�ˡA�N�O�N���Ѥѷݹ�A�ܤ��n���N��I)�C

��ı�o�p�G�ڭ̩��a�@�ӨS���e�~�B�S���d�Y�B�S���IJv���Ʒ~�b�@�ɤW�~��s�b�ۡA����Ѥѷݤ@�w�|�ܥͮ�I�o�N�O�ҿת��u�Ѹݡv�C�ܩ�Ѥѷݷ|���A�����ij�O�H���G�ܤ����A�Ѥѷݵ���|�i�D�A�O�Q�Ӧh�A�N�u�M���v�a�I

�n�a�I�ڴN�g��o�̤F�A�o�N�O�F�~���ѭn��j�a���ɪ��_�L�y�p�G�ơA�H�Ϋ�ҳ\�[���u�ѸݺM���סv�C�o�O�ڳ̪����޿�W���@�Ӥj��}�A�Ʊ�]�i�H�a���U��Ū�̪B�ͤ@�I���F�i���C

�ܩ�ڨ��Ӷi���~�h�B�J���w�B�]��ժ��B�ӥB�D�`��J�A���O�����o���B�D�`�S�����z���S�IJv�Ʒ~�C�b�_�L�y�j�v����ij�U�A�ڤ]���D�ӫ�򰵤F�I�`���A�ڤ��Q�Q�I���N�u��F�I
 
   % 所附的範例
%\input{intro.tex}
%\input{experiment.tex}
%\input{theory.tex}
%\input{calculation.tex}
%\input{summary.tex}

% back pages 後頁
% 包括參考文獻、附錄、自傳
% 實際內容由
%    my_bib.bib, my_appendix.tex, my_vita.tex
% 決定
% yzu_backpages.tex 此檔只提供整體架構的定義,不需更動
% 在撰寫各章草稿時,可以把此部份「關掉」,以節省無謂的編譯時間。
%
% this file is encoded in big5
% v2.01 (Jul. 3, 2012)

%%% �ѦҤ��m
\newpage
\phantomsection % for hyperref to register this
\addcontentsline{toc}{chapter}{\nameRef}
\renewcommand{\bibname}{\protect\makebox[5cm][s]{\nameRef}}
%  \makebox{} is fragile; need protect
\bibliographystyle{IEEEtran}  % �ϥ� IEEE Trans ���Z�榡
\bibliography{my_bib}


%%% ����
%
% this file is encoded in utf-8
% v2.02 (Sep. 12, 2012)
%%% 每一個附錄 (附錄甲、附錄乙、...) 都要複製此段附錄編排碼做為起頭
%%% 附錄編排碼 begin >>>
\newpage 
\chapter*{附錄 A:  MATLAB / Octave 程式列表} % 修改附錄編號與你的附錄名
\phantomsection % for hyperref to register this
\addcontentsline{toc}{chapter}{附錄 A: MATLAB / Octave 程式列表} %建議此內容應與上行相同
%\setcounter{chapter}{0}  %如果用的是 TeXLive2007 則打開此行以避免錯誤 
\setcounter{equation}{0} 
\setcounter{figure}{0} 
\setcounter{footnote}{0} 
\setcounter{section}{0} 
\setcounter{subsection}{0}
\setcounter{subsubsection}{0}
\setcounter{table}{0} 
\renewcommand{\thechapter}{A} % 如果是附錄 B,則內容應為{B}
%%% <<< 附錄編排碼 end

% 附錄內容開始
%%% 納入程式源碼
\lstinputlisting[caption={matlab 程式碼列表範例},
label=lst:matlab:example,
numbers=left,
firstnumber=1,
frame=ltrb, % single lines for left, top, right, bottom; LTRB for double lines 
escapeinside={$$}, %如要在列表裡顯示特殊字元/排版效果,要把該文字串用 $$ 包夾住 (適合 C 程式碼)(原預設為 <>)
]
{example_prog_list.m}

\begin{equation}\sum_{k=1}^{n} k = \frac{n(n+1)}{2}\end{equation}

%%% 如果有附錄B、C、...,則在此繼續加上「附錄編排」碼
% 每一個附錄會自動以新頁開始

%%% �۶�
\newpage
\chapter*{\protect\makebox[5cm][s]{\nameVita}} % \makebox{} is fragile; need protect
\phantomsection % for hyperref to register this
\addcontentsline{toc}{chapter}{\nameVita}
���H�ͩ� 1981 �~ 1 �� 1 ��A�b��餺�c�C�a�̸g��q����A�W���@��n�n�C�q�p�N���w��ѩ��̦��^�����o�a�q�Ϋ~�A�m�N�F�@���n�����P���s�@�����n�_�ߡC

��p�NŪ������p�C�ѩ��������եΤ���������F��}��s�ˤ��^�h�A�y�����հ����A�Z�Ҧ�©�����C�Q�ժ��B�@���Z�Ҥ@�ӬP���C���u�O�ڤ֮ɦ~���L�����@�������C



