%
% this file is encoded in utf-8
% v2.02 (Sep. 12, 2012)
%%% 每一個附錄 (附錄甲、附錄乙、...) 都要複製此段附錄編排碼做為起頭
%%% 附錄編排碼 begin >>>
\newpage 
\chapter*{附錄 A:  MATLAB / Octave 程式列表} % 修改附錄編號與你的附錄名
\phantomsection % for hyperref to register this
\addcontentsline{toc}{chapter}{附錄 A: MATLAB / Octave 程式列表} %建議此內容應與上行相同
%\setcounter{chapter}{0}  %如果用的是 TeXLive2007 則打開此行以避免錯誤 
\setcounter{equation}{0} 
\setcounter{figure}{0} 
\setcounter{footnote}{0} 
\setcounter{section}{0} 
\setcounter{subsection}{0}
\setcounter{subsubsection}{0}
\setcounter{table}{0} 
\renewcommand{\thechapter}{A} % 如果是附錄 B,則內容應為{B}
%%% <<< 附錄編排碼 end

% 附錄內容開始
%%% 納入程式源碼
\lstinputlisting[caption={matlab 程式碼列表範例},
label=lst:matlab:example,
numbers=left,
firstnumber=1,
frame=ltrb, % single lines for left, top, right, bottom; LTRB for double lines 
escapeinside={$$}, %如要在列表裡顯示特殊字元/排版效果,要把該文字串用 $$ 包夾住 (適合 C 程式碼)(原預設為 <>)
]
{example_prog_list.m}

\begin{equation}\sum_{k=1}^{n} k = \frac{n(n+1)}{2}\end{equation}

%%% 如果有附錄B、C、...,則在此繼續加上「附錄編排」碼
% 每一個附錄會自動以新頁開始