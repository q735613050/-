%
% this file is encoded in utf-8
% v2.03 (Dec. 19, 2012)

%%% ZZZ %%% 如果不是用 LaTeX CJKnumb 套件,
%% 則必須自己製作陽春的 \CJKnumber
%% 請把下列介於 >>> 與 <<< 之間
%% 的文字行打開 (行首移除百分號)
%% 中文數字對應指令 (只能對應 1 到 10)
%% 參考自 LaTeX-CJK doc: CJK.txt, zh-Hant.cpx
%% 用法:\CJKnumber{1} 可得一

%%% 陽春 \CJKnumber 命令定義 >>>
%\makeatletter
%\@ifundefined{CJKnumber}
%  {\def\CJKnumber#1{\ifcase#1\or一\or二\or三\or四\or五\or%
% 六\or七\or八\or九\or十\fi}}{}
%\makeatother
%%% <<< 陽春 \CJKnumber 命令定義

%%%%%%%%%%%%%%%%%%%%%%%%%%%
%%% 天干對應 \CJKordinal 
\makeatletter
\@ifundefined{CJKordinal}
  {\def\CJKordinal#1{\ifcase#1\or甲\or乙\or丙\or丁\or戊\or%
 己\or庚\or辛\or壬\or癸\fi}}{}
\makeatother
%%

% 名詞 \prechaptername 預設值為 'Chapter '
% 名詞 \postchaptername 預設值為空字串

%%%%%%%%%%%%%%%%%%%%%%%%%%%%%%%%%%%%


% 請依需要選擇其中一種表現方式,把它所對應的指令列打開,其他沒有用到的表現方式的對應指令列請關閉。(用行首百分號)

%% 第一種目錄格式:
%%	1  簡介 ............................ 1
%%
%%      章別 (chapter counter) 「1」前後沒有其他文字,
%%
%%      內文章標題是
%%		第 1 章	簡介
%%	\tocprechaptername, \tocpostchaptername 都設成沒有內容的空字串
%%	\tocChNumberWidth 設成 1.4em (預設)
%%      底下三行指令請打開
%\renewcommand\tocprechaptername{}
%\renewcommand\tocpostchaptername{}
%\setlength{\tocChNumberWidth}{1.4em}


%% 第二種目錄格式:
%%	一、簡介 ............................ 1
%%
%%      章別 (chapter counter) 「一」前沒有文字,後有頓號,
%%
%%      內文章標題是
%%		第一章		簡介
%%	\tocprechaptername 設成沒有內容的空字串
%%	\tocpostchaptername 設成頓號
%%	\tocChNumberWidth 設成 2em
%%      底下六行指令請打開 (預設)
\renewcommand\countermapping[1]{\CJKnumber{#1}}
\renewcommand\tocprechaptername{}
\renewcommand\tocpostchaptername{、}
\setlength{\tocChNumberWidth}{2em}
\renewcommand\prechaptername{第} % 出現在每一章的開頭的「第x章」, x前後沒有空白
\renewcommand\postchaptername{章}


%% 第三種目錄格式:
%%	第一章、簡介 ......................... 1
%%
%%      章別 (chapter counter) 「一」前有「第」,後有「章」與頓號,
%%      內文章標題是
%%		第一章		簡介
%%	\tocprechaptername 設成「第」
%%	\tocpostchaptername 設成「章、」
%%	\tocChNumberWidth 設成 3em
%%      底下六行指令請打開
%\renewcommand\countermapping[1]{\CJKnumber{#1}}
%\renewcommand\tocprechaptername{第}
%\renewcommand\tocpostchaptername{章、}
%\setlength{\tocChNumberWidth}{3em}
%\renewcommand\prechaptername{第} % 出現在每一章的開頭的「第x章」, x前後沒有空白
%\renewcommand\postchaptername{章}


%% 第四種目錄格式:
%%	1  簡介 ............................ 1
%%
%%      章別 (chapter counter) 「1」前後沒有其他文字,
%%
%%      內文章標題是
%%		Chapter 1	簡介
%%	\tocprechaptername, \tocpostchaptername 都設成沒有內容的空字串
%%	\tocChNumberWidth 設成 1.4em (預設)
%%      底下五行指令請打開
%\renewcommand\tocprechaptername{}
%\renewcommand\tocpostchaptername{}
%\setlength{\tocChNumberWidth}{1.4em}
%\renewcommand\prechaptername{Chapter } % 出現在每一章的開頭的「Chapter x」, x前有空白
%\renewcommand\postchaptername{}


%% 可以依照需要作彈性的設定
%%
%% 在目錄裡的章別 (數字,包括後面的字串) 的寬度 \tocChNumberWidth,
%% 會影響章名與章別之間的間隔 (太少則相疊,太多則留白)
%% 建議設成 \tocpostchaptername 內容字數加一,做為 em 的倍數,
%% 但至少也要有 1.4 倍。
