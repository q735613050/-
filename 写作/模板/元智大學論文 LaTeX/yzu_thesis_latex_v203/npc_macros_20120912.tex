%% non-float version of table and figure environments
\makeatletter
\newenvironment{tablehere}
  {\def\@captype{table}}
  {}

\newenvironment{figurehere}
  {\def\@captype{figure}}
  {}
\makeatother


% some useful macros
%

% a single figure
%
% usage:
% \fig{width}
% {path/filename}
% {caption}
% {label}
\newcommand{\fig}[4]{
\begin{figure}[tbph]
{\centering \includegraphics[%
  width=#1,%
  keepaspectratio]{#2}
 \caption{#3}
 \label{#4}
 \par
}
\end{figure}
}

% a single figure, with additional text beneath the caption.  Usually the source of the figure.
%
% usage:
% \figt{width}
% {path/filename}
% {caption}
% {label}
% {additional text beneath the caption.  Usually the source of the figure.}
\newcommand{\figt}[5]{
\begin{figure}[tbph]
{\centering \includegraphics[%
  width=#1,%
  keepaspectratio]{#2}
 \caption{#3}
 \label{#4}
 #5\par
 }
\end{figure}
}

% single figure, additional text below the caption, spacing control between the figure and the caption
%
% usage:
% \figts{width}
% {path/filename}
% {caption}
% {label}
% {additional text beneath the caption.  Usually the source of the figure.}
% {place negative height, say, -4ex to reduce the gap between fig and caption}
\newcommand{\figts}[6]{
\begin{figure}[tbph]
{\centering \includegraphics[%
  width=#1,%
  keepaspectratio]{#2}
  \vspace{#6}\par
 \caption{#3}
 \label{#4}
 #5\par
 }
\end{figure}
}

% single figure, additional text below the caption, spacing control between the figure and the caption, trimming
%
% usage:
% \figtsr{width}
% {path/filename}
% {caption}
% {label}
% {additional text beneath the caption.  Usually the source of the figure.}
% {place negative height, say, -4ex to reduce the gap between fig and caption}
% {left bottom right top} in unit of bp (=1/72 in)
\newcommand{\figtsr}[7]{
\begin{figure}[tbph]
{\centering \includegraphics[%
  width=#1,%
  keepaspectratio,%
  trim=#7,clip=true]{#2}
  \vspace{#6}\par
 \caption{#3}
 \label{#4}
 #5\par
 }
\end{figure}
}

% multiple figures and captions in side-by-side arrangement;
%
% usage: (must be within figure environment)
% \mpfig{width}% in terms of \columnwidth
% {path/filename}
% {caption}
% {label}
\newcommand{\mpfig}[4]{
\begin{minipage}[b][1\totalheight]{#1\columnwidth}
\par\vspace{0pt}
{\centering \includegraphics[width=1\columnwidth,keepaspectratio]{#2}
\caption{#3}
\label{#4}\par}
\end{minipage}
}

% multiple figures and captions in side-by-side arrangement;
% each has its own descriptive text beneath the caption.
%
% usage: (must be within figure environment)
% \mpfigt{width}% in terms of \columnwidth
% {path/filename}
% {caption}
% {label}
% {additional text beneath the caption.  Usually the source of the figure.}
\newcommand{\mpfigt}[5]{
\begin{minipage}[b][1\totalheight]{#1\columnwidth}
{\centering \includegraphics[width=1\columnwidth,keepaspectratio]{#2}
\caption{#3}
\label{#4}
#5\par}
\end{minipage}
}

% multiple figures and captions in side-by-side arrangement, with spacing gap control;
%
% usage: (must be within figure environment)
% \mpfigts{width}% in terms of \columnwidth
% {path/filename}
% {caption}
% {label}
% {additional text beneath the caption.  Usually the source of the figure.}
% {place negative height, say, -4ex to reduce the gap between fig and caption}
\newcommand{\mpfigts}[6]{
\begin{minipage}[b][1\totalheight]{#1\columnwidth}
\par\vspace{0pt}
{\centering \includegraphics[width=1\columnwidth,keepaspectratio]{#2}
\vspace{#6}\par
\caption{#3}
\label{#4}
#5\par}
\end{minipage}
}

% multiple figures and captions in side-by-side arrangement, with spacing, w/ trim;
%
% usage: (must be within figure environment)
% \mpfigtra{width}% in terms of \columnwidth
% {path/filename}
% {caption}
% {label}
% {additional text beneath the caption.  Usually the source of the figure.}
% {place negative height, say, -4ex to reduce the gap between fig and caption}
% {left bottom right top} in terms of bp (=1/72 in)
\newcommand{\mpfigtsr}[7]{
\begin{minipage}[b][1\totalheight]{#1\columnwidth}
\par\vspace{0pt}
{\centering \includegraphics[width=1\columnwidth,keepaspectratio,%
 trim=#7,clip=true]{#2}
\vspace{#6}\par
\caption{#3}
\label{#4}
#5\par}
\end{minipage}
}

% multiple figures and captions in side-by-side arrangement, w/ spacing, w/ trim, w/ alignment;
%
% usage: (must be within figure environment)
% \mpfigtsra{width}% in terms of \columnwidth
% {path/filename}
% {caption}
% {label}
% {additional text beneath the caption.  Usually the source of the figure.}
% {place negative height, say, -4ex to reduce the gap between fig and caption}
% {left bottom right top}
% {alignment}
\newcommand{\mpfigtsra}[8]{
\begin{minipage}[#8][1\totalheight]{#1\columnwidth}
\par\vspace{0pt}
{\centering \includegraphics[width=1\columnwidth,keepaspectratio,%
 trim=#7,clip=true]{#2}
\vspace{#6}\par
\caption{#3}
\label{#4}
#5\par}
\end{minipage}
}

% multiple figures in side-by-side arrangement, but with a single caption
%
% usage: (must be within figure environment)
%     the caption capability is not included in this macro
%
% \mpfigabc{width}% in terms of \columnwidth
% {path/filename}
% {text beneath the figure.  Usually for the numbering of the subfigure}
\newcommand{\mpfigabc}[3]{
\begin{minipage}[b][1\totalheight]{#1\columnwidth}
\par\vspace{0pt}
{\centering \includegraphics[width=1\columnwidth,keepaspectratio]{#2}
#3\par}
\end{minipage}
}

% multiple figures in side-by-side arrangement, but with a single caption, w/spacing
%
% usage: (must be within figure environment)
%     the caption capability is not included in this macro
%
% \mpfigabcs{width}% in terms of \columnwidth
% {path/filename}
% {text beneath the figure.  Usually for the numbering of the subfigure}
% {place negative height, say, -4ex to reduce the gap between fig and caption}
\newcommand{\mpfigabcs}[4]{
\begin{minipage}[b][1\totalheight]{#1\columnwidth}
\par\vspace{0pt}
{\centering \includegraphics[width=1\columnwidth,keepaspectratio]{#2}\par
\vspace*{#4}#3\par}
\end{minipage}
}

% multiple figures in side-by-side arrangement, but with a single caption, w/spacing, w/trimming
%
% usage: (must be within figure environment)
%     the caption capability is not included in this macro
%
% \mpfigabcsr{width}% in terms of \columnwidth
% {path/filename}
% {text beneath the figure.  Usually for the numbering of the subfigure}
% {place negative height, say, -4ex to reduce the gap between fig and caption}
% {left bottom right top}
\newcommand{\mpfigabcsr}[5]{
\begin{minipage}[b][1\totalheight]{#1\columnwidth}
\par\vspace{0pt}
{\centering \includegraphics[width=1\columnwidth,keepaspectratio,%
 trim=#5,clip=true]{#2}\par
\vspace*{#4}#3\par}
\end{minipage}
}



% a single figure; FIGUREHERE version
%
% usage:
% \hfig{width}
% {path/filename}
% {caption}
% {label}
\newcommand{\hfig}[4]{
\begin{figurehere}
{\centering \includegraphics[%
  width=#1,%
  keepaspectratio]{#2}
\caption{#3}
\label{#4}
\par
}
\end{figurehere}
}

% a single figure, with additional text beneath the caption.  Usually the source of the figure. FIGUREHERE version
%
% usage:
% \hfigt{width}
% {path/filename}
% {caption}
% {label}
% {additional text beneath the caption.  Usually the source of the figure.}
\newcommand{\hfigt}[5]{
\begin{figurehere}
{\centering \includegraphics[%
  width=#1,%
  keepaspectratio]{#2}%
\caption{#3}
\label{#4}
#5\par}
\end{figurehere}
}

% controlling the gap between fig and caption
% FIGUREHERE version
%
% usage:
% \hfigts{width}
% {path/filename}
% {caption}
% {label}
% {additional text beneath the caption.  Usually the source of the figure.}
% {place negative height, say, -4ex to reduce the gap between fig and caption}
\newcommand{\hfigts}[6]{
\begin{figurehere}
{\centering \includegraphics[%
  width=#1,%
  keepaspectratio]{#2}
\vspace{#6}\par
\caption{#3}
\label{#4}
#5\par}
\end{figurehere}
}

% single figure, additional text below the caption spacing control between the figure and the caption, trimming
% FIGUREHERE version
%
% usage:
% \hfigtsr{width}
% {path/filename}
% {caption}
% {label}
% {additional text beneath the caption.  Usually the source of the figure.}
% {place negative height, say, -4ex to reduce the gap between fig and caption}
% {left bottom right top} in unit of bp (= 1/72 in)
\newcommand{\hfigtsr}[7]{
\begin{figurehere}
{\centering \includegraphics[%
  width=#1,%
  keepaspectratio,%
  trim=#7,clip=true]{#2}
\vspace{#6}\par
\caption{#3}
\label{#4}
#5\par
}
\end{figurehere}
}

