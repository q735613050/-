%
% this file is encoded in utf-8
% v2.02 (Sep. 12, 2012)

\documentclass[landscape,12pt]{article}

% 如果使用 latex + dvipdfmx 工作流程,要特別使用 -l 來指定 landscape 排版
%   dvipdfmx -l yzu_coverbridge.dvi
%
% pdflatex 工作流程則自動選定 landscape 排版
%
% 需要 LaTeX CJK 套件 4.6.0 或更新的版本
% 
\usepackage{CJKutf8}
\usepackage{CJKspace}

\usepackage{geometry}
\geometry{verbose,a4paper,landscape, tmargin=1cm, bmargin=1cm, lmargin=1.5cm, rmargin=1.5cm}


\ifx\pdfoutput\undefined
  % not running pdftex
  \usepackage[dvipdfmx]{graphicx}
\else
  \ifx\pdfoutput\relax
    % not running pdftex
    \usepackage[dvipdfmx]{graphicx}
  \else
    % running pdftex, with...
    \ifnum\pdfoutput>0
      % ... PDF output
      \usepackage[pdftex]{graphicx}
    \else
      %...DVI output
      \usepackage[dvipdfmx]{graphicx}
    \fi
  \fi
\fi


\usepackage{rotating} %旋轉封面 minipage 與「畢業年份」阿拉伯數字所需之套件
\usepackage{CJKvert} %旋轉中文字所需之套件

\parindent = 0Pt  % 段首內縮由 CJK 控制,所以這裡就設成不內縮

\begin{document}
\thispagestyle{empty} % no page number
%
\begin{CJK*}{UTF8}{bkai} 
%所選之中文字體須有配合直排之 .fdx 檔,否則所穿插的英、數符號與中文字會對不齊
%
% varible place holders
% 此處只是預設值,不需更改此處
% 請更改 my_names.tex 內容

\newcommand\cTitle{}
\newcommand\eTitle{}
\newcommand\myCname{}
\newcommand\myEname{}
\newcommand\advisorCnameA{}
\newcommand\advisorEnameA{}
\newcommand\advisorCnameB{}
\newcommand\advisorEnameB{}
\newcommand\advisorCnameC{}
\newcommand\advisorEnameC{}
\newcommand\univCname{}
\newcommand\univEname{}
\newcommand\deptCname{}
\newcommand\fulldeptEname{}
\newcommand\deptEname{}
\newcommand\collEname{}
\newcommand\degreeCname{}
\newcommand\degreeEname{}
\newcommand\cYear{}
\newcommand\cMonth{}
\newcommand\eYear{}
\newcommand\eMonth{}
\newcommand\ePlace{}

% 取得個人資料
%
% this file is encoded in utf-8
% v2.0 (Apr. 5, 2009)
% 填入你的論文題目、姓名等資料
% 如果題目內有必須以數學模式表示的符號,請用 \mbox{} 包住數學模式,如下範例
% 如果中文名字是單名,與姓氏之間建議以全形空白填入,如下範例
% 英文名字中的稱謂,如 Prof. 以及 Dr.,其句點之後請以不斷行空白~代替一般空白,如下範例
% 如果你的指導教授沒有如預設的三位這麼多,則請把相對應的多餘教授的中文、英文名
%    的定義以空的大括號表示
%    如,\renewcommand\advisorCnameB{}
%          \renewcommand\advisorEnameB{}
%          \renewcommand\advisorCnameC{}
%          \renewcommand\advisorEnameC{}

% 論文題目 (中文)
\renewcommand\cTitle{%
我的碩士論文題目 \mbox{$\cal{H}_\infty$} 與 \mbox{Al$_x$Ga$_{1-x}$As}
}

% 論文題目 (英文)
\renewcommand\eTitle{%
My Thesis Title  \mbox{$\cal{H}_\infty$} and \mbox{Al$_x$Ga$_{1-x}$As}
}

% 我的姓名 (中文)
\renewcommand\myCname{王鐵雄}

% 我的姓名 (英文)
\renewcommand\myEname{Aron Wang}

% 指導教授A的姓名 (中文)
\renewcommand\advisorCnameA{南宮明博士}

% 指導教授A的姓名 (英文)
\renewcommand\advisorEnameA{Dr.~Ming Nangong}

% 指導教授B的姓名 (中文)
\renewcommand\advisorCnameB{李斯坦博士}

% 指導教授B的姓名 (英文)
\renewcommand\advisorEnameB{Dr.~Stein Lee}

% 指導教授C的姓名 (中文)
\renewcommand\advisorCnameC{徐 石博士}

% 指導教授C的姓名 (英文)
\renewcommand\advisorEnameC{Dr.~Sean Hsu}

% 校名 (中文)
\renewcommand\univCname{元智大學}

% 校名 (英文)
\renewcommand\univEname{Yuan Ze University}

% 系所名 (中文)
\renewcommand\deptCname{光電工程學系}

% 系所全名 (英文)
\renewcommand\fulldeptEname{Department of Electro-Optical Engineering}

% 系所短名 (英文, 用於書名頁學位名領域)
\renewcommand\deptEname{Electro-Optical Engineering}

% 學院英文名 (如無,則以空的大括號表示)
\renewcommand\collEname{College of Electrical and Communication Engineering}

% 學位名 (中文)
\renewcommand\degreeCname{碩士}

% 學位名 (英文)
\renewcommand\degreeEname{Master of Science}

% 口試年份 (中文、民國)
\renewcommand\cYear{九十八}

% 口試月份 (中文)
\renewcommand\cMonth{七} 

% 口試年份 (阿拉伯數字、西元)
\renewcommand\eYear{2009} 

% 口試月份 (英文)
\renewcommand\eMonth{July}

% 學校所在地 (英文)
\renewcommand\ePlace{Chungli, Taoyuan, Taiwan}

%畢業級別;用於書背列印;若無此需要可忽略
\newcommand\GraduationClass{97}

%%%%%%%%%%%%%%%%%%%%%%

%%%%%%%%%% 封面
\begin{sideways} % rotate the minipage ccw-90 deg
\begin{minipage}[b][26.2cm]{13cm} 
% A4=21cm x 29.7cm; cover margin: t=3.5cm, b=2cm, l=4cm, r=2cm
% the cover minipage dimension: width = 21-2-4 = 15 cm; height = 29.7 - 3.5 - 2 = 24.2cm
% the left edge of the cover minipage is 4cm away from the cover bridge (left margin)
% now, we have 1cm A4 landscape bottom margin (left side of the cover), 1cm cover bridge width, 4cm gap
% there is no space for A4 landscape top margin (right side of the cover)
% we have to reduce the width of the cover minipage
% the widest box in the cover is the department name (12 cm)
% so, we reduce the cover minipage width to 13 cm
% Since all boxes in the cover mnipage is center aligned,
% the reduced minipage has now 5cm gap with the coverbridge. 
%
% About the height of the minipage, the theoretical value is 24.2cm. There is a mysterious bumper of 2cm;
% as a result, the last line (the date) is 2cm higher. So, the height of the minipage is set to 24.2 + 2 = 26.2cm
%
\CJKhorz %to avoid un-necessary rotation of CJK characters in cover minipage
%
\vspace*{2cm} 
% the landscape A4 left margin=1.5cm; the "top" of the cover (left of landscape A4) needs additional 2cm to achieve the 3.5cm top margin of the cover
%
%%%%%%%%%%%% Duplicate the code for cover from yzu_frontpages.tex
% aligned to the center of the page
\begin{center}
% font size (relative to 12 pt):
% \large (14pt) < \Large (18pt) < \LARGE (20pt) < \huge (24pt)< \Huge (24 pt)
%
\makebox[6cm][s]{\Huge\univCname}\\  %顯示中文校名
\vspace{1.5cm}
\makebox[12cm][s]{\Huge\deptCname}\\ %顯示中文系所名
\vspace{1.5cm}
\makebox[6cm][s]{\Huge\degreeCname 論文}\\ %顯示論文種類 (中文)
\vspace{1.5cm}
%
% Set the line spacing to single for the titles (to compress the lines)
\renewcommand{\baselinestretch}{1}   %行距 1 倍
%\large % it needs a font size changing command to be effective
\Large\cTitle\\  % 中文題目
%
\vspace{1cm}
%
\Large\eTitle\\ %英文題目
\vspace{2cm}
% \makebox is a text box with specified width;
% option s: stretch; option l: left aligned
% use \makebox to make sure
% 「研究生」 與「指導教授」occupy the same width
% Names are filled in a box with pre-defined width
% the left and right sides of 「:」occupy the same width (use \hspace{} to fill the short)
% to guarantee 「:」is at the center
% assume the width of a Chinese character is 1.2em
% 4.8em is determined by the length of the longest string "指導教授"
% 7.2em is determined by the length of the possibly longest name + title "歐陽明志博士"
\hspace{2.4em}%
\makebox[4.8em][s]{\Large 研究生}%
\makebox[1em][c]{\Large :}%
\makebox[7.2em][l]{\Large\myCname}\\  % 顯示作者中文名
%
\hspace{2.4em}%
\makebox[4.8em][s]{\Large 指導教授}%
\makebox[1em][c]{\Large :}%
\makebox[7.2em][l]{\Large\advisorCnameA}\\  %顯示指導教授A中文名
%
% 判斷是否有共同指導的教授 B
\ifx \advisorCnameB  \itsempty
\relax % 沒有 B 教授,所以不佔版面,不印任何空白
\else
% 共同指導的教授 B
\hspace{2.4em}%
\makebox[4.8em][s]{}%
\makebox[1em][c]{}%
\makebox[7.2em][l]{\Large\advisorCnameB}\\%顯示指導教授B中文名
\fi
%
% 判斷是否有共同指導的教授 C
\ifx \advisorCnameC  \itsempty
\relax % 沒有 C 教授,所以不佔版面,不印任何空白
\else
% 共同指導的教授 C
\hspace{2.4em}%
\makebox[4.8em][s]{}%
\makebox[1em][c]{}%
\makebox[7.2em][l]{\Large\advisorCnameC}\\%顯示指導教授B中文名
\fi
%
\vfill
\makebox[10cm][s]{\Large 中華民國\cYear 年\cMonth 月}%
%
\end{center}
\end{minipage}
\end{sideways}
% Resume the normal font size (12 pt)

\normalsize

\vspace{5cm} % gap between the cover minipage and the cover bridge
%%%%%%%%%%%%

%
% 中文直排 (反時鐘旋轉每一中文字 90 度)
\CJKvert
%
% font size:
% \large (14pt) < \Large (18pt) < \LARGE (20pt) < \huge (24pt)< \Huge (24 pt)
%
% 重新定義中文字距,避免旋轉後之中文字過密
\renewcommand{\CJKglue}{\hskip 0.5pt plus 0.08\baselineskip}
%
%
%
% 開始排版
%
%
% 口試年份 (畢業級別)
% 中線下降 2.5 pt (兩位數的級別,如 97, 98, 99)
% 中線下降 4 pt (三位數的級別,如 100,...)
\raisebox{-4pt}{%
\makebox[1cm][l]{%
\begin{sideways}\GraduationClass\end{sideways}%
}
}%
%
% 顯示論文種類
\makebox[2.5cm][c]{\Large\degreeCname 論文}%
%
\hfill
% 論文篇名 (如果篇名太長,會自動折行)
% 因為旋轉後的中文,在 makebox 與 parbox 裏的基線有些微差距 (原因不明),使得以 parbox 編排的論文篇名與校所名會比其他以 makebox 編排的部分低 (直立後偏左)。所以用 raisebox 微調
\newcommand\myFineTune{1.75pt} % 微調偏移量
%
\raisebox{\myFineTune}{%
\parbox{13.5cm}{\renewcommand\baselinestretch{0.5}\Large\cTitle}%
}
\hfill
%
% 校、所名
\raisebox{\myFineTune}{%
\parbox{3.1cm}{% 校名與系所名要很靠近,所以用 \tiny 小號字體取得小行距
\tiny \makebox[3cm][s]{\small\univCname}\\%
\makebox[3cm][s]{\footnotesize\deptCname}%
}%
}%
%
%
\hspace{0.8cm}
%
\makebox[2cm][s]{\Large\myCname}%
%
\end{CJK*}
\end{document} 
